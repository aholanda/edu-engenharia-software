\input{../quiz-header}

\title{Teste de ``Engenharia de Software I''}
\author{prof. Adriano J. Holanda}
\date{13/3/2015}

\begin{document}

\quizheader{FAFRAM}{Sistemas de Informação}

\quiz~{\bf Não} é elemento de um diagrama de fluxo de dados:

\begin{enumerate}[a)]
\item depósito de dados
\item fluxo
\item classe
\item entidade externa
\item processo
\end{enumerate}

\end{document}

%%%%%%%%%%%%%%%%%%%%%%%%%%%%%%%% METHODS

\quiz {\bf Não} constitue uma das fases de desenvolvimento de sistemas:

\begin{enumerate}[a)]
\item implementação;
\item levantamento de requisitos;
\item implantação;
\item projeto;
\item cobrança.
\end{enumerate}

\quiz {\bf Não} constitui fase do método em cascata (clássico):

\begin{enumerate}[a)]
\item Análise de requisitos;
\item Implementação;
\item Projeto;
\item Prototipagem;
\item Implantação.
\end{enumerate}

\quiz O método em espiral incorpora características do cascata e evolutivo com 
a adição de:

\begin{enumerate}[a)]
\item Programação em pares;
\item Prototipação;
\item Análise de risco;
\item Validação;
\item Testes.
\end{enumerate}

\quiz Com relação aos métodos de produção do software é {\bf incorreto} afirmar:

\begin{enumerate}[a)]
\item Os métodos ágeis dão ênfase à documentação do sistema;
\item O método em espiral enfatiza a importância da análise de riscos;
\item No método em cascata normalmente não há construção de protótipos;
\item No método evolutivo, a construção de protótipos é essencial;
\item O levantamento de requisitos é a fase inicial dos métodos em
  cascata, evolutivo e espiral.
\end{enumerate}


