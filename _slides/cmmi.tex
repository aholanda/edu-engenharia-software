\lecture{CMMI: \em Capability Maturity Model Integration}{intro}
\lecturetitle{\insertlecture}{\course}

\frame{\maketitle}

\begin{frame}{\insertlecture}{1987}

Iniciativa do Departamento de Defesa (DoD) e Instituto de Engenharia
de Software da Universidade de Carnegie Mellon, ambos nos Estados
Unidos para estabelecer um modelo de produção de software com as
seguintes propostas:

\begin{itemize}[<+->]
\item Ser baseado em experiência prática das empresas de software;
\item Refletir melhor o estado da prática;
\item Ser documentado e estar disponível publicamente;
\item Atender as necessidades daqueles que realizam melhorias no processo de produção de software.
\end{itemize}

\end{frame}

\begin{frame}{Características de uma empresa imatura}
\footnotesize
O objetivo do CMMI é tornar uma empresa madura com relação aos
 processos de produção de software. Uma empresa imatura possui uma ou
 mais das seguintes características:

\begin{itemize}[<+->]
\item O trabalho é feito em regime de emergência (apagar incêndio);
\item Dificilmente os compromissos de prazo e custo são cumpridos;
\item Não é usual fazer planejamento com base em estimativas realistas;
\item Como os processos não são bem definidos, eventuais iniciativas 
      de melhoria não se sustentam ou não se perpetuam;
\item Quando o projeto é pressionado por prazo, características de 
      qualidade e funcionalidade do produto são sacrificadas;
\item O sucesso de um projeto depende de alguns poucos especialistas (gurus), 
      que resolvem todos os grandes problemas ou lançam mão de novas 
      tecnologias como solução milagrosa.
\end{itemize}

\end{frame}

\lecture{Níveis de Capacidade e Maturidade}{level}

\section{\insertlecture}

\onlytitleframe{\insertlecture}

\begin{frame}{\insertlecture}
\footnotesize
Comparação dos níveis de capacidade e maturidade.\\\bigskip
\begin{tabular}{l|l|l}\hline
\bf Nível &\bf Representação contínua dos  &\bf Representação discreta dos  \\
& \bf níveis de capacidade & \bf níveis de maturidade\\ \hline\hline
Nível 0 & Incompleto & \\\hline
Nível 1 & Realizado & Inicial \\\hline
Nível 2 & Gerenciado & Gerenciado \\\hline
Nível 3 & Definido & Definido \\\hline
Nível 4 & & Gerenciado quantitativamente \\\hline
Nível 5 & & Em Otimização \\\hline
\end{tabular}
\end{frame}

\begin{frame}{Níveis de capacidade}\small
\begin{description}[<+->]
\item[0. Incompleto:] Um  processo incompleto é um processo que
  não é realizado ou parcialmente realizado.
\item[1. Realizado:] Um processo realizado é um processo que atinge o
  objetivo de produção. Apesar de ser um avanço, as melhorias são
  perdidas se não for uma política institucional.
\item[2. Gerenciado:] Um processo gerenciado é planejado e executado
  de acordo com uma política; emprega pessoas capacitadas para
  produzir resultados controlados, seguindo a sequência de descrição
  do processo.
\item[3. Definido:] Um processo definido segue as métricas da
  organização de acordo com alguma norma. Produz experiência a partir
  do relato e medidas da sequência de descrição do
  processo. Possibilita análises qualitativas e quantitativas.
\end{description}
\end{frame}

\begin{frame}{Níveis de maturidade}\footnotesize
\begin{description}[<+->]
\item[1. Inicial:] No nível de maturidade 1, os processos são
  normalmente {\em ad~hoc} e caóticos. A organização não propicia um
  ambiente estável de suporte aos processos.
\item[2. Gerenciado:] Os processos são planejados e executados de
  acordo com uma política. Similar ao nível de capacidade 2.
\item[3. Definido:] Os processos são bem caracterizados e entendidos,
  e são descritos em normas, procedimentos, ferramentas e métodos. Os
  processos são mais bem descritos e normatizados que o nível 2.
\item[4. Gerenciado Quantitativamente:] A organização estabelece
  objetivos quantitativos para a qualidade e desempenho do processo e
  os utiliza no gerenciamento do processo. É utilizada análise
  estatística para monitoramento e controle dos processos.
\item[5. Em Otimização:] A organização aperfeiçoa continuamente seus
  processos baseada no entendimento quantitativo dos objetivos do
  negócio e necessidades de desempenho.
\end{description}
\end{frame}

\lecture{Áreas de processo no CMMI}{areas}
\section{\insertlecture}
\onlytitleframe{\insertlecture: gerenciamento de processos,
  gerenciamento de projetos, engenharia, suporte.}

\begin{frame}{Gerenciamento de processos}
  \begin{itemize}[<+->]
  \item Definição de processo organizacional;
  \item Foco de processo organizacional;
  \item Treinamento organizacional;
  \item Desempenho do processo organizacional;
  \item Inovação e implantação organizacional.
  \end{itemize}
\end{frame}

\begin{frame}{Gerenciamento de projetos}
  \begin{itemize}[<+->]
  \item Planejamento de projeto;
  \item Monitoramento e controle de projeto;
  \item Gerenciamento de acordo com fornecedores;
  \item Gerenciamento de projeto integrado;
  \item Gerenciamento de riscos;
  \item Gerenciamento quantitativo de projeto.
  \end{itemize}
\end{frame}

\begin{frame}{Engenharia}
  \begin{itemize}[<+->]
  \item Gerenciamento de requisitos;
  \item Desenvolvimento de requisitos;
  \item Solução técnica;
  \item Integração de produto;
  \item Verificação;
  \item Validação.
  \end{itemize}
\end{frame}

\begin{frame}{Suporte}
  \begin{itemize}
  \item Gerenciamento de configuração;
  \item Garantia de qualidade de processo e produto;
  \item Medição e análise;
  \item Análise de decisão e resolução;
  \item Análise causal e resolução.
  \end{itemize}
\end{frame}

\begin{frame}{Alcance das metas}
  \begin{itemize}
  \item {\bf Monitoramento e controle do processo:} 
    \begin{itemize}
    \item Ações corretivas são gerenciadas até a conclusão quando o
      desempenho ou os resultados do projeto se desviam
      significativamente do plano.
    \item O desempenho real e o progresso do projeto são monitorados
      de acordo com o plano do projeto.
    \end{itemize}
  \item {\bf Desenvolvimento de requisitos:} Os requisitos são
    analisados e validados, e uma definição da funcionalidade
    requerida é desenvolvida.
  \item {\bf Análise causal e resolução:} Causa-raiz de defeitos e
    outros problemas são determinados.
  \end{itemize}
\end{frame}

\section{Referências}

\begin{frame}{Referências}
  \begin{itemize}
  \item CMMI Product Team. ``CMMI for Development'', Version
    1.3. Technical Report, 2010. Disponível em
    \hbox{\footnotesize\url{http://goo.gl/pchLoz}}.
  \item \ianref{}
  \item Mário Lúcio Côrtes; Thelma C. dos Santos Chiossi. ``Modelos de
    Qualidade de Software''. Editora da UNICAMP, 2001.
  \end{itemize}
\end{frame}