\lecture{Levantamento e Análise de Requisitos}{requirements}

\lecturetitle{\course}{\insertlecture}

\section{\insertlecture}

\frame{\maketitle}

\begin{frame}{\insertlecture}
  
  \begin{itemize}[<+->]\setbeamercovered{transparent}
  \item O software deve se adaptar às necessidades dos usuários;
    \begin{itemize}
    \item As melhores técnicas de programação---algoritmos, estrutura de dados, 
      análise de performance, estrutura modular---não salvarão o projeto da falha.
    \end{itemize}
  \item Os softwares são construídos para atender os clientes, em especial os usuários, 
    e devem se adaptar às suas necessidades.
  \end{itemize}

  \pause
  
  O levantamento e análise de requisitos tenta atingir um equilíbrio
  entre o que os usuários querem e o que o sistema fará.
  \note{Equilíbrio porque alguma negociação deve haver.}

\end{frame}

\begin{frame}{Produtos da fase requisitos}
  \begin{description}[<+->]\setbeamercovered{transparent}
  \item[Documento de requisitos] descrevendo as características do
    software a ser construído.
  \item[Plano de validação] descrevendo como o futuro software, uma
    vez construído, será testado.
  \end{description}
\end{frame}

\begin{frame}{O Padrão IEEE}
  \begin{itemize}
  \item
    \href{http://holanda.xyz/files/720574.pdf}{Prática
      Recomendada para as Especificações de Requisitos de
      Software. IEEE Computer Society. IEEE Std 830-1998.}

    \begin{itemize}
    \item Conjunto de listas para checagem de propriedades que o sistema
      deve satisfazer.
    \end{itemize}
    \note{O padrão ajuda a levantar as propriedades, tentando antecipar
      problemas que poderiam ocorrer.}
  \end{itemize}
\end{frame}

\begin{frame}{Escopo dos Requisitos}
  \begin{description}[<+->]\setbeamercovered{transparent}
  \item[Requisitos Funcionais] definem as funções que o sistema deve
    possuir. Possuem entrada, processamento e saída.
  \item[Requisitos Não-Funcionais] performance, portabilidade, éticos,
    legais, integração.
  \end{description}
\end{frame}

\begin{frame}{Obtendo os Requisitos}
  
  A obtenção dos requisitos é uma {\bf negociação} entre as propriedades
  desejadas pelo usuário para o sistema e as possibilidades de
  desenvolvimento e custos expostas pela equipe técnica.

  \pause

  \begin{block}{Principais fatores a serem equilibrados}
  \begin{itemize}[<+->]\setbeamercovered{transparent}
  \item Alguns usuários têm visões conflitantes sobre o sistema;
  \item Algumas propriedades desejadas são inviáveis em termos de custo ou 
    desenvolvimento;
  \item Os usuários tendem a pensar em termos de sistemas existentes;
  \item Fatores externos afetam a escolha das funcionalidades,
    principalmente, custo e interferência da instituição onde o
    sistema será implantado.
  \end{itemize}
\end{block}

\end{frame}

\begin{frame}{Técnicas para levantamento de requisitos}
  \begin{itemize}[<+->]\setbeamercovered{transparent}
  \item Entrevistas;
  \item Workshops;
  \item Sistemas anteriores;
  \item Sistemas similares.
  \end{itemize}
\end{frame}

\frame{\author{}\date{}\title{Documentação de requisitos}\maketitle}

\subsection{Documentação de requisitos}

\begin{frame}{Documentação dos requisitos}
  Formas de escrever uma especificação de requisitos do sistema:

  \begin{description}[<+->]\setbeamercovered{transparent}
  \item[Sentenças em linguagem natural:] Os requisitos são escritos em forma de frases.
  \item[Linguagem natural estruturada:] Os requisitos são escritos em um formulário.
    \item[Notações gráficas:] Utiliza figuras e diagramas com texto associado.
  \item[Especificações matemáticas:] Usa notações matemáticas, como máquina de estados 
    finitos e teoria de conjuntos.
  \end{description}
\end{frame}

\begin{frame}{Sentenças em linguagem natural}
  Exemplo de requisitos funcionais de um sistema de gestão de pessoas, disciplinas e notas 
  de uma Faculdade ou Universidade:
  \begin{enumerate}
  \item Cadastro de pessoas por funcionários da Faculdade.
    \item Consulta de pessoas por funcionários da Faculdade.
    \item Cadastro de disciplinas por funcionário da Faculdade. 
    \item Inserção de notas por professores.
    \item Consulta de notas por parte dos alunos.
    \end{enumerate}
\end{frame}

\begin{frame}{Linguagem natural estruturada}

  {\bf Sistema de gerenciamento para Faculdade ou Universidade}\\\bigskip

  \begin{tabular}[h]{|l|l|}\hline
    \bf Função: & Gerenciar dados de pessoas, disciplinas e notas.\\\hline
    \bf Descrição: & O sistema deve armazenar os dados de funcionários, \\
    & alunos e docentes, de disciplinas e de notas.\\\hline
    \bf Entradas: & Pessoas: nome, endereço, RG, CPF.\\
                & Disciplinas: nome, responsável, resumo.\\
                & Notas: nome da disciplina, nome do aluno, valor.\\\hline
    \bf Saídas: & Visualização das entradas.\\\hline
    
  \end{tabular}
\end{frame}

\begin{frame}{Notação gráfica}
  
  Exemplo de uso de notação gráfica para a documentação de requisitos, no 
  caso, diagrama de casos de uso da UML ({\em Unified Modeling Language}):

  \begin{center}
    \includegraphics[scale=.35]{usecase.png}
  \end{center}

\end{frame}

\begin{frame}{Especificações matemáticas}
  
  Especificação do sistema de uma catraca usando máquina de estados finitos:\bigskip
\begin{center}
  \includegraphics[scale=.35]{state.png}
\end{center}
\end{frame}

\begin{frame}[fragile]{Especificações matemáticas, continuação}

  Especificação de um relógio binário usando a lógica temporal de ações (TLA$^+$, \href{http://research.microsoft.com/en-us/um/people/lamport/tla/tla.html}{\em Temporal Logic of Actions}):

\begin{multline}
  \text{VARIABLE clock}\hfill\\
  \text{Init} \equiv clock \in \{0, 1\}\hfill\\
  \text{Tick} \equiv \text{IF}\ clock = 0\ \text{THEN}\ clock' = 1\ \text{ELSE}\ clock' = 0\hfill\\
  \text{Spec} \equiv \text{Init} \land [][\text{Tick}]_{\langle clock\rangle}\hfill
\end{multline}

\end{frame}

\begin{frame}{Referências}  
  \begin{enumerate}
  \item \ianref
  \item \pfref
  \item Bertrand Meyer, ``Touch of Class: Learning to Program Well
    with Objects and Contracts'', Springer, 2009.
  \end{enumerate}
\end{frame}