\lecturetitle{\em Refactoring, cont.}{\course}

\frame{\maketitle}

\lstset{language=java,basicstyle=\scriptsize,commentstyle=\tiny\color{gray}}

% \begin{frame}{\em Refactoring}
% \begin{itemize}[<+- | alert@+>]\setbeamercovered{transparent}
% \item {\em Refactoring} é o processo de reescrever material escrito para
%     aperfeiçoar sua legibilidade ou estrutura sem o propósito explícito de
%     alterar seu significado ou comportamento.
%   \item Teve maior divulgação com o livro ``Refactoring: Improving the
%     Design of Existing Code'' (ISBN 0-201-48567-2) de Martin Fowler, e de
%     devido a sua associação às Metodologias Ágeis, principalmente
%     \emph{Extreme Programming}.
%   \end{itemize}
% \end{frame}

% \ifnum1=2

% \begin{frame}{\em Encapsulate Field}

%   \begin{columns}
%     \begin{column}{.5\textwidth}
%       \begin{block}{Antes}
%         \onslide<1>{\color{gray}}
%         \begin{center}
%           \begin{tikzpicture}
%             \begin{class}[text width=8cm]{ Pessoa }{0 ,0}
%               \attribute {+ nome : String}
%             \end{class}
%           \end{tikzpicture}
%         \end{center}
%       \end{block}
%     \end{column}

%     \begin{column}{.5\textwidth}
%       \begin{block}{Antes}
%         \begin{center}
%           \begin{tikzpicture}
%             \begin{class}[text width=8cm]{ Pessoa }{0 ,0}
%               \attribute{- nome : String}

%               \operation{+ getNome() : String}
%               \operation{+ setNome(nome : String)}
%             \end{class}
%           \end {tikzpicture}
%         \end{center}
%       \end{block}
%     \end{column}
%   \end{columns}

% \end{frame}


% \begin{frame}{\em Extract Class}\footnotesize
% \begingroup
%     \onslide<1>{\color{gray}}
%     \begin{center}
%   \begin{tikzpicture}
%     \begin{class}[text width=6cm]{Pessoa}{0 ,0}
%       \attribute {- nome : String}
%       \attribute {- noFone : String}
%     \end{class}
%   \end{tikzpicture}
% \end{center}
% \endgroup

% \pause\bigskip

%     \begin{center}
%     \begin{tikzpicture}
%       \begin{class}[text width=4cm]{Pessoa}{-3 ,-.5}
%         \attribute{- nome : String}
%     \end{class}

%     \begin{class}[text width=4cm]{Telefone}{3,0}
%       \attribute{- DDD : String}
%       \attribute{- no : String}
%       \attribute{- ramal : String}
%     \end{class}
%     \unidirectionalAssociation{Pessoa}{possui}{0..*}{Telefone}
%   \end {tikzpicture}
% \end{center}

% \end{frame}

% \fi


%%%%%%%%%%%%%%%%%%%%%%%%%%%%%%%% FIELD/METHOD

 \begin{frame}{\em Push Down Field/Method}\small
   \label{Upushdownlabel}
  \begin{center}
    \begin{tikzpicture}
      \begin{class}[text width=4cm]{Funcionario}{0,2.5}
        \attribute{- comissao : float}
        \operation{+ getComissao() : float}
      \end{class}
      \begin{class}[text width=3cm]{Vendedor}{-3,0}
        \inherit{Funcionario}
      \end{class}
      \begin{class}[text width=3cm]{Engenheiro}{3,0}
        \inherit{Funcionario}
      \end{class}
    \end{tikzpicture}
  \end{center}

  \vfill Inverso: {\em Push Up Field/Method}, slide~\ref{Upushuplabel}.
\end{frame}

\begin{frame}[fragile]{\em Push Down Field/Method}{\alert{\em refactoring}}\small
  \label{pushdownlabelU}
  \begin{center}
    \begin{tikzpicture}
      \begin{class}[text width=3cm]{Funcionario}{0,2.5}
      \end{class}
      \begin{class}[text width=4cm]{Vendedor}{-3,0}
        \inherit{Funcionario}
        \attribute{- comissao : float}
        \operation{+ getComissao() : float}
      \end{class}
      \begin{class}[text width=3cm]{Engenheiro}{3,0}
        \inherit{Funcionario}
      \end{class}
    \end{tikzpicture}
  \end{center}
    \vfill Inverso: {\em Push Up Field/Method}, slide~\ref{pushuplabelU}.
\end{frame}


\begin{frame}{\em Push Up Field/Method}\small
  \label{Upushuplabel}
  \begin{center}
    \begin{tikzpicture}
      \begin{class}[text width=3cm]{Funcionario}{0,2}
      \end{class}
      \begin{class}[text width=4cm]{Vendedor}{-3,0}
        \attribute{- nome : String}
        \operation{+ getNome() : String}
        \inherit{Funcionario}
      \end{class}
      \begin{class}[text width=4cm]{Engenheiro}{3,0}
        \attribute{- nome : String}
        \operation{+ getNome() : String}
        \inherit{Funcionario}
      \end{class}
    \end{tikzpicture}
  \end{center}

  \vfill Inverso: {\em Push Down Field/Method}, slide~\ref{Upushdownlabel}.
\end{frame}

\begin{frame}{\em Push Up Field/Method}{\alert{\em refactoring}}\small
  \label{pushuplabelU}

  \begin{center}
    \begin{tikzpicture}
      \begin{class}[text width=4cm]{Funcionario}{0,2.5}
        \attribute{\#  nome : String}
        \operation{+ getNome() : String}
      \end{class}
      \begin{class}[text width=3cm]{Vendedor}{-3,0}
        \inherit{Funcionario}
      \end{class}
      \begin{class}[text width=3cm]{Engenheiro}{3,0}
        \inherit{Funcionario}
      \end{class}
    \end{tikzpicture}
  \end{center}

    \vfill Inverso: {\em Push Down Field/Method}, slide~\ref{pushdownlabelU}.
\end{frame}

%%%%%%%%%%%%%%%%%%%%%%%%%%%%%%%% DLEIF


%%%%%%%%%%%%%%%%%%%%%%%%%%%%%%%% METHOD BODY

\begin{frame}[fragile]{\em Replace Temp with Query}
 \begin{lstlisting}[language=java]
public class _Produto {
        private int quantidade;
        private double precoItem;

        public double calcularPrecoTotal() {
                double precoBase = this.quantidade * this.precoItem;
                if (precoBase > 1000) {
                        return precoBase * 0.95;
                } else {
                        return precoBase * 0.98;
                }
        }
}
  \end{lstlisting}
\end{frame}


\begin{frame}[fragile]{\em Replace Temp with Query}{\em\color{red}refactoring}
\color{red}
  \begin{lstlisting}[language=java]
public class Produto_ {
        private int quantidade;
        private double precoItem;

        public double calcularPrecoTotal() {
                if (this.precoBase() > 1000) {
                        return this.precoBase() * 0.95;
                } else {
                        return this.precoBase() * 0.98;
                }
        }

        private double precoBase() {
                return this.quantidade*this.precoItem;
        }
}
  \end{lstlisting}
\end{frame}

\begin{frame}[fragile]{\em Move Method}\footnotesize

  \begin{columns}
    \begin{column}{.5\textwidth}
      \begin{block}<1->{Antes}\bigskip
        \begin{tikzpicture}
          \begin{class}[text width=4.6cm]{Projeto}{0 ,0}
            \attribute{- participantes: Pessoa[]}
          \end{class}

          \begin{class}[text width=4.6cm]{Pessoa}{0,-2}
            \attribute{- id : int}
            \operation{+ participa(p: Projeto) : boolean}
          \end{class}
          %\unidirectionalAssociation{Pessoa}{possui}{0..*}{Telefone}
        \end {tikzpicture}
      \end{block}
    \end{column}
    
    \begin{column}{.5\textwidth}
      \begin{block}<2>{Depois}\bigskip
        \begin{tikzpicture}
          \begin{class}[text width=4.6cm]{Projeto}{0,0}
            \attribute{- participantes: Pessoa[]}
            \operation{+ participa(p: Pessoa) : boolean}
          \end{class}
          
          \begin{class}[text width=4.6cm]{Pessoa}{0,-2.5}
            \attribute{- id : int}
          \end{class}
          %\unidirectionalAssociation{Pessoa}{possui}{0..*}{Telefone}
        \end {tikzpicture}
      \end{block}
    \end{column}
  \end{columns}
\end{frame}



\begin{frame}[fragile]{\em Extract Method}
\begin{lstlisting}[language=java]
  public float calcularSalario(float salario, int numAnos) {
    int adicional;
    // Calcular o adicional por anos trabalhados,
    // 1) a cada 5 anos +5% sobre o salario;
    // 2) e a cada 20 anos +8% sobre o salario.
    adicional = 5 * (numAnos / 5);
    adicional += 8 * (numAnos/20);

    return salario + salario * ((float)adicional/100);
  }
\end{lstlisting}

\pause

\begin{lstlisting}[basicstyle=\tiny\color{red}]
  public float calcularSalario(float salario, int numAnos) {
    int adicional = calcularAdicional(numAnos);
    
    return salario + salario * ((float)adicional/100);
   }
   // Calcula o adicional por anos trabalhados,
   // 1) a cada 5 anos +5% sobre o salario;
   // 2) e a cada 20 anos +8% sobre o salario.
public int calcularAdicional(int numAnos) {
    int adicional = 5 * (numAnos / 5);
    adicional += 8 * (numAnos/20);
    
    return adicional;
  }
\end{lstlisting}
\end{frame}

\begin{frame}{Exercício}
  \begin{enumerate}
  \item Implementar o exemplo \emph{Extract Method} em Java.
  \end{enumerate}
\end{frame}

%%%%%%%%%%%%%%%%%%%%%%%%%%%%%%%% END OF METHOD BODY