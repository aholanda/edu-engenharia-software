\lecturetitle{Automação da Montagem do Sistema}{\course}

\frame{\maketitle}

\section{Automação da Montagem do Sistema}

\begin{frame}{Automação da Montagem}
 A automação da montagem do sistema permite que tarefas como
compilação e vinculação de bibliotecas dinâmicas, instalação do
software e até mesmo implantação sejam realizadas sem a intervenção do
desenvolvedor, seguindo regras de execução e ordem de dependências.

As ferramentas mais comuns para a realização destas tarefas são:

\begin{itemize}
\item \href{https://www.gnu.org/software/make/}{GNU make}
\item \href{https://cmake.org/}{CMake}
\item \href{http://ant.apache.org/}{Apache Ant}
\item \href{https://github.com/microsoft/msbuild}{Microsoft.Build}
\end{itemize}
\end{frame}

\begin{frame}[fragile]{GNU make}
 Exemplo de arquivo de configuração para o \href{https://www.gnu.org/software/make/}{GNU make}.\bigskip

{\tt Makefile}
\hrule\small
\begin{verbatim}
 objects = main.o kbd.o command.o display.o \
          insert.o search.o files.o utils.o

 edit : $(objects)
        cc -o edit $(objects)

$(objects) : defs.h

kbd.o command.o files.o : command.h

display.o insert.o search.o files.o : buffer.h
\end{verbatim}

\tiny
Fonte: \url{https://www.gnu.org/software/make/}
\end{frame}

\begin{frame}[fragile]{Apache Ant}

Utilizado principalmente em projetos na linguagem Java. Segue exemplo 
de arquivo de configuração:\bigskip
\scriptsize
{\tt build.xml}
\hrule\tiny
\begin{verbatim}
<project>
    <target name="clean">
        <delete dir="build"/>
    </target>

    <target name="compile">
        <mkdir dir="build/classes"/>
        <javac srcdir="src" destdir="build/classes"/>
    </target>

    <target name="jar">
        <mkdir dir="build/jar"/>
        <jar destfile="build/jar/HelloWorld.jar" basedir="build/classes">
            <manifest>
                <attribute name="Main-Class" value="oata.HelloWorld"/>
            </manifest>
        </jar>
    </target>

    <target name="run">
        <java jar="build/jar/HelloWorld.jar" fork="true"/>
    </target>

</project>
\end{verbatim}
\tiny Fonte: \url{http://ant.apache.org/}
\end{frame}


