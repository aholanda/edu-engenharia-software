\title{Gerenciamento de Configuração}

\frame{\maketitle}

\begin{frame}{\inserttitle}{Definição}

\inserttitle{} se refere a um conjunto de procedimentos para 
controlar a evolução de um sistema/software. Tipicamente inclui 
as seguintes tarefas:

\begin{itemize}[<+-| alert@+>]
\item Controle de versão;
\item Automação da montagem do sistema;
\item Automação dos testes e rastreio de {\em bugs} ({\em
  bug-tracking});
\item Controle do lançamento de versões.
\end{itemize}

\end{frame}


\lecture{Automação de testes}{tests}
\section{\insertlecture}
\onlytitleframe{\insertlecture}

%%%%%%%%%%%%%%%%%%%%%%%%%%%%%%%% CONTINUOUS INTEGRATION %%%%%%%%%%%%%%%%
% Fontes
% http://martinfowler.com/articles/continuousIntegration.html
% http://www.yegor256.com/2014/10/08/continuous-integration-is-dead.html

\lecture{Integação Contínua}{ci}
\section{\insertlecture}
\onlytitleframe{\insertlecture}

\begin{frame}{\insertlecture}
  
  \begin{itemize}[<+->]\setbeamercovered{transparent}
  \item Prática de desenvolvimento de software.
  \item Membro da equipe envia suas modificações para o repositório central.
  \item No repositório, cada envio é seguido pela montagem e testes.
  \item O erro durante a integração exige a atenção da equipe para corrigí-lo.
  \end{itemize}
  
\end{frame}

\begin{frame}{\insertlecture}{Práticas}

  \begin{itemize}[<+->]\setbeamercovered{transparent}
  \item Manter um repositório único para o projeto.
  \item Automatizar a montagem.
  \item Fazer com que a montagem teste a si própria.
  \item A equipe envia as modificações para a árvore principal 
    do controle de versão todo dia.
  \item Os erros devem ser reparados imediatamente.
  \item Qualquer um pode ter acesso ao executável mais recente.
  \item Todos podem ver o que está ocorrendo com a implantação 
    automatizada.
  \end{itemize}
  
\end{frame}

\begin{frame}{\insertlecture}{Benefícios e Riscos}

  \begin{description}[<+->]\setbeamercovered{transparent}
  \item[Benefícios] 
    \begin{itemize}
    \item Auxilia a identificação de {\em bugs} precocemente.
    \item Maior confiança para a adição de funcionalidades.
    \item Permite o acesso à versão mais atual do software.
    \end{itemize}
  \item[Riscos] 
    \begin{itemize}
    \item Se a montagem falha, a equipe para para reparar o 
      erro, mesmo aqueles que não estão relacionados com a 
      falha.
    \end{itemize}
  \end{description}

\end{frame}

\begin{frame}{\insertlecture}{Programas}

  \begin{itemize}
  \item \href{https://continuum.apache.org/}{Apache Continuum};
  \item \href{http://hudson-ci.org/}{Hudson};
  \item \href{http://cruisecontrol.sourceforge.net/}{CruiseControl};
  \item \href{https://jenkins.io/}{Jenkins};
  \item \href{https://www.appveyor.com/}{AppVeyor} (web).
  \end{itemize}

\end{frame}

\lecture{Gerenciamento de Bugs}{bugs}
\section{\insertlecture}
\onlytitleframe{\insertlecture}

\begin{frame}{Sistema de Gerenciamento de Bugs}
\begin{itemize}[<+-| alert@+>]
\item O {\bf Sistema de Gerenciamento de {\em Bugs}} ({\em Bug Tracking System})  é uma aplicação que centraliza o relato, análise e solução de {\em bugs} encontrados no software desenvolvido ou em desenvolvimento.
\item É interessante que o gerenciamento de {\em bugs} esteja vinculado à 
ferramenta onde é feito o gerenciamento do projeto, para vinculação com 
código fonte, cronograma, documentação e outros objetos que auxiliam no 
desenvolvimento e manutenção do software.
\item Deve possuir sistema de notificações por email, autenticação, 
 sistema de permissões, segurança e confidencialidade.
\end{itemize}
\end{frame}

\begin{frame}{Algumas Implementações}{Sistema de Gerenciamento de Bugs}

\begin{itemize}
\item \href{https://www.bugzilla.org/}{Bugzilla}
\item \href{https://www.atlassian.com/software/jira}{JIRA}
\item \href{https://www.mantisbt.org/}{Mantis}
\item \href{http://www.redmine.org/}{Redmine}
\item \href{http://trac.edgewall.org/}{Trac}
\end{itemize}

\end{frame}
