
\begin{frame}{Pontos de Função}
  A Análise de Pontos de Função (APF) é uma métrica elaborada por
  Allan Albrecht da IBM,
  \href{http://fattocs.com/files/pt/artigos/MedindoaProdutividadedoDesenvolvimentodeAplicativos.pdf}{publicada}
  em 1979.\pause\bigskip

  O padrão reconhecido pela indústria de software para APF é o Manual
  de Práticas de Contagem de Pontos de Função (CPM - Counting
  Practices Manual) mantido pelo IFPUG ({\em International Function
    Point Users Group}).

\end{frame}

\begin{frame}{Funções de Dados}{Pontos de Função}
  \small
  \begin{description}[<+->]\setbeamercovered{transparent}
  \item[Arquivos Lógicos Internos (ALI):] {\em Internal Logic File (ILF)}
    \begin{itemize}
    \item Entidade lógica e persistente;
    \item Mantida dentro da fronteira da aplicação;
    \item Equivalente ao repositório de dados.
    \end{itemize}

  \item[Arquivos de Interface Externa (AIE):] {\em External Interface File (EIF)}
    \begin{itemize}
    \item Entidade lógica e persistente;
    \item Mantida dentro da fronteira de outra aplicação;
    \item Equivalente a um {\em WebService}, RPC, CORBA.
    \end{itemize}
  \end{description}
\end{frame}


\begin{frame}{Funções de Transações}\small
\begin{description}[<+->]\setbeamercovered{transparent}
  \item[Entrada Externa (EE):] {\em External Input (EI)}
    \begin{itemize}
    \item Processo lógico do negócio que mantem os dados;
    \item Contada com base no número de campos de dados.
    \end{itemize}

  \item[Saída Externa (SE):] {\em External Output (EO)}
    \begin{itemize}
    \item Processo lógico do negócio que gera dados;
    \item Exemplos: Relatórios, emails de resposta.
    \end{itemize}

  \item[Consulta Externa (CE):] {\em External Query (EQ)}
    \begin{itemize}
    \item Processamento sem alteração de estado.
    \end{itemize}
  \end{description}
\end{frame}

\begin{frame}{Complexidade}\footnotesize

  \begin{itemize}[<+->]\setbeamercovered{transparent}
  \item ALI, AIE\\
    \begin{center}
    \begin{tabular}[ht]{|c|c|c|c|}\hline
      & \bf <  20 & \bf 20--50 &\bf  >50 \\\hline
      1 & Baixa & Baixa & Média \\
      2--5 & Baixa &  Média & Alta \\
      >5 & Média & Alta & Alta \\\hline
    \end{tabular}
  \end{center}
  
  \item EE\\
    \begin{center}
    \begin{tabular}[ht]{|c|c|c|c|}\hline
      & \bf <  5 & \bf 5--15 &\bf  >15 \\\hline
      <2 & Baixa & Baixa & Média \\
      2 & Baixa &  Média & Alta \\
      >2 & Média & Alta & Alta \\\hline
    \end{tabular}
  \end{center}

\item SE, CE\\
    \begin{center}
    \begin{tabular}[ht]{|c|c|c|c|}\hline
      & \bf <  6 & \bf 6--19 &\bf  >19 \\\hline
      <2 & Baixa & Baixa & Média \\
      2--3 & Baixa &  Média & Alta \\
      >3 & Média & Alta & Alta \\\hline
    \end{tabular}
  \end{center}
\end{itemize}
\end{frame}

\begin{frame}{Pesos}
  
  A contribuição de cada complexidade é extraída da tabela a seguir:

  \begin{center}
    \begin{tabular}[ht]{|c|c|c|c|}\hline
      \bf Tipo & \bf Baixa & \bf Média & \bf Alta \\\hline
      ALI & 7 & 10 & 15 \\
      AIE & 5 & 7 & 10 \\
      EE & 3 & 4 & 6 \\
      SE & 4 & 5 & 7 \\
      CE & 3 & 4 & 6 \\\hline
    \end{tabular}
  \end{center}
\end{frame}

\begin{frame}{Fator de Ajuste do Valor}\scriptsize
  
  \begin{columns}
    \begin{column}{.3\textwidth}
      Níveis de Influência
      \smallskip
      \begin{tabular}[ht]{|c|c|}\hline
        \bf Valor & Influência \\\hline
        0 & Nenhuma \\
        1 & Mínima \\
        2 & Moderada \\
        3 & Média \\
        4 & Significativa \\
        5 & Forte \\\hline
      \end{tabular}
    \end{column}
    \begin{column}{.7\textwidth}
      \begin{tabular}[ht]{l|l}\hline\hline
        \bf \hfil Características  & \bf\hfil  Nível  de\\
        \bf \hfil Gerais & \bf \hfil Influência \\\hline
        1. Comunicação de Dados & \\
        2. Processamento Distribuído de Dados & \\
        3. Desempenho & \\
        4. Configuração Intensamente Utilizada& \\
        5. Taxa de Transação& \\
        6. Entrada de Dados On-Line& \\
        7. Eficiência do Usuário Final& \\
        8. Atualização On-Line& \\
        9. Processamento Complexo& \\
        10. Reutilização& \\
        11. Facilidade de Instalação& \\
        12. Facilidade de Operação& \\
        13. Múltiplas Localidades& \\
        14. Facilidade de Alteração& \\\hline
        \bf TOTAL & \\\hline\hline
      \end{tabular}
    \end{column}
  \end{columns}

  \begin{center}
    Fator de Ajuste  = Nível de Influência Total * 0,01 + 0,65
  \end{center}
\end{frame}

\begin{frame}{Ponto de Função Ajustado}
  
  \begin{center}
    Ponto de Função Ajustado = Ponto de Função não Ajustado * Fator de Ajuste
  \end{center}
\end{frame}

\begin{frame}{Referências}
  \begin{itemize}
  \item
    \href{http://www.slideshare.net/alvarofpinheiro/medida-de-esforo-de-software-com-anlise-de-ponto-de-funo}{Medida
      de Esforço de Software com Análise de Ponto de Função}.  Álvaro
    Farias Pinheiro. [slides]
  \item
    \href{http://www.bfpug.com.br/Artigos/Dekkers-PontosDeFuncaoEMedidas.htm}{Pontos
      de Função e Medidas. O que é um Ponto de Função?} Carol
    A. Dekkers, Quality Plus Technologies, Inc.
  \end{itemize}
  
\end{frame}