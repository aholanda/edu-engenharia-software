%%% Local Variables: 
%%% mode: latex
%%% TeX-master: "main"
%%% End: 

\title{UML}

\frame{\includegraphics{img/uml.png}\maketitle}

\begin{frame}{UML}{Introdução}
  
  \begin{itemize}
  \item UML - {\it Unified Modeling Language} -- Linguagem Unificada
    de Modelagem.\pause
  \item Adquiriu maturidade na segunda década de 1990 pela fusão dos
    métodos e diagramas de
    \href{https://en.wikipedia.org/wiki/Grady_Booch}{Grady Booch}, 
    \href{https://en.wikipedia.org/wiki/Ivar_Jacobson}{Ivar H. Jacobson} e
    \href{https://en.wikipedia.org/wiki/James_Rumbaugh}{James E. Rumbaugh}.\pause
  \item Incorporada pela \href{http://www.ibm.com/software/rational}{Rational} 
    Software, comprada pela IBM.\pause
  \item Em 1997, foi adotada como padrão pela
    \href{http://www.omg.org/}{OMG}~({\it Object Management Group}).\pause
  \item Em 2005, foi publicada pela
    \href{http://www.iso.org/iso/home.html}{ISO}~({\it International
      Organization for Standardization}).
  \end{itemize}

\end{frame}

\begin{frame}{UML}{Termos e conceitos}
  \begin{description}
  \item[Sistema:] coleção de subsistemas organizados para a realização
    de um objetivo.\pause
  \item[Subsistema] agrupamento de elementos, alguns dos quais
    constituem uma especificação do comportamento.\pause
  \item[Modelo:] abstração semanticamente fechada de um sistema.\pause
  \item[Visão:] projeção da organização e estrutura de um modelo de um
    sistema.
  \end{description}

\end{frame}

\begin{frame}{UML}{Diagramas: parte estática}
  A \alert{estrutura (parte estática)} de um sistema tipicamente pode ser
  visualizada pela utilização dos seguintes diagramas:

  \begin{description}
  \item[Classes:] classes, interfaces, colaborações e seus relacionamentos.\pause
  \item[Objetos:] conjunto de objetos e seus relacionamentos.\pause
  \item[Componentes:] conjunto de componentes e seus relacionamentos.\pause
  \item[Pacotes:] descreve as dependências entre os pacotes que formam um modelo.\pause
  \item[Implantação:] conjunto de nós e seus componentes.
  \end{description}

\end{frame}

\begin{frame}{UML}{Diagramas: parte dinâmica}
  O \alert{comportamento (parte dinâmica)} de um sistema tipicamente pode ser
  visualizada pela utilização dos diagramas a seguir:

  \begin{description}
  \item[Caso de uso:] organiza os comportamentos do sistema.\pause
  \item[Sequência:] tem como foco a ordem temporal das mensagens.\pause
  \item[Colaboração:] tem como foco a organização estrutural de objetos que enviam 
    e recebem mensagens.\pause
  \item[Gráfico de estados:] Focaliza o estado de mudança de um
    sistema orientado por eventos.\pause
  \item[Atividades:] Focaliza o fluxo de controle de um atividade para outra.\pause
  \item[Comunicação:] na UML 2.0, é a versão simplificada do diagrama de 
    colaboração da UML 1.x.
  \end{description}

\end{frame}

\begin{frame}{Regras da UML}
  
  A UML dispõe de regras semânticas para a construção de modelos
  \alert{bem-formados}, que são auto-consistentes semanticamente e em
  harmonia com todos os modelos a eles relacionados. A UML dispõe de
  regras para:

  \begin{description}
  \item[Nomes:] quais nomes podem ser atribuídos a coisas,
    relacionamentos e diagramas.\pause
  \item[Escopo:] o contexto que determina um significado específico
    para um nome.\pause
  \item[Visibilidade:] como esses nomes podem ser vistos utilizados 
    pelos outros.\pause
  \item[Integridade] como os itens se relacionam entre si de forma 
    adequada e consistente.\pause
  \item[Execução:] o que significa executar ou simular um modelo
    dinâmico.
  \end{description}
  
\end{frame}

\begin{frame}{Arquitetura}
  
  A arquitetura é o conjunto de decisões significativas acerca dos
  seguintes itens:

  \begin{itemize}
  \item A organização do sistema de software.\pause
  \item A seleção dos elementos estruturais e suas interfaces, que
    compõe o sistema.\pause
  \item Seu comportamento, conforme especificado nas colaborações 
    entre esses elementos.\pause
  \item A composição desses elementos estruturais e comportamentais 
    em subsistemas progressivamente maiores.\pause
  \item O estilo de arquitetura que orienta a organização: os elementos 
    estáticos e dinâmicos e suas respectivas interfaces, colaborações 
    e composições.
  \end{itemize}

\end{frame}

\begin{frame}[fragile]{Referência}
  \begin{thebibliography} {Rumbaugh, 1998}

  \bibitem[Rumbaugh, 1998]{rumbaugh1998}
    James Rumbaugh, Ivar Jacobson, Grady Booch.
    \newblock 	   {\em The Unified Modeling Language User Guide}.
    \newblock	   Addison-Wesley, 1998.

  \end{thebibliography}
\end{frame}
