%% Local variables:
% TeX-master: main
% TeX-engine: lualatex
%% End:

\title{Diagrama de Classes\\UML ({\em Unified Modeling Language})}

\frame{\maketitle}

\begin{frame}[fragile]{Nome da Classe}

  \begin{itemize}
  \item  Nome da entidade a ser representada pela classe.
  \end{itemize}

  \begin{center}
    \begin{tikzpicture}
      \begin{class}{NomeClasse}{0,0}
      
      \end{class}
    \end{tikzpicture}
  \end{center}

\end{frame}


\begin{frame}[fragile]{Atributos da Classe}{\only<1>{Conceitual}\only<2>{Especificação}}
  \begin{itemize}
  \item  Propriedades escolhidas para descrever a abstração da entidade.
  \end{itemize}

  \begin{center}
    \begin{tikzpicture}
      \onslide<1> {
        \begin{class}{Pessoa}{0,0}
          \attribute{nome}
          \attribute{endereco}
          \attribute{fone}
          \attribute{dataNascimento}
        \end{class}
      }

      \onslide<2> {
        \begin{class}{Pessoa}{0,0}
          \attribute{nome : String}
          \attribute{endereco : String}
          \attribute{fone : String}
          \attribute{dataNascimento : Date}
        \end{class}
      }
    \end{tikzpicture}
  \end{center}

\end{frame}

\begin{frame}[fragile]{Operações ou Métodos}{\only<1>{Conceitual}\only<2>{Especificação}}
  \begin{itemize}
  \item  Especificação de ações que modificam o estado da classe ou apenas realizam algum
    procedimento.
  \end{itemize}

  \begin{center}
    \begin{tikzpicture}

      \onslide<1>{
        \begin{class}{Quadrado}{0,0}
          \operation{redimensionar()}
          \operation{mover()}
          \operation{preencher()}
          \operation{comprimento()}
        \end{class}
      }

      \onslide<2>{
        \begin{class}[text width=8cm]{Quadrado}{0,0}
          \operation{redimensionar(comprimento : Integer )}
          \operation{mover(ponto : Point)}
          \operation{preencher(cor: Color)}
          \operation{comprimento() : Integer}
        \end{class}
      }
                                  
    \end{tikzpicture}
  \end{center}

\end{frame}

\begin{frame}[fragile]{Interface}
  \begin{itemize}
  \item Não possui operações implementadas.
  \end{itemize}

  \begin{center}
    \begin{tikzpicture}
      \begin{interface}{IPessoa}{0,0}
        \attribute{nome : String}
        \attribute{sobrenome : String}
        \attribute{dataNascimento : Date}
        \operation{calcularIdade() : Integer}
      \end{interface}
    \end{tikzpicture}
  \end{center}

  \bigskip{A operação {\tt calcularIdade()} não possui implementação, serve como contrato de 
    nomes de operações a serem implementadas.}

\end{frame}

\begin{frame}[fragile]{Classe Abstrata}
  \begin{itemize}
  \item Possui algumas operações implementadas.
  \end{itemize}

  \begin{center}
    \begin{tikzpicture}
      \begin{abstractclass}{ContaBancaria}{0,0}
        \attribute{numero : Integer}
        \attribute{correntista : String}
        \attribute{saldo : Currency = 0}
    
        \operation{depositar(valor : Currency)}
        \operation{\it calcularSaldo() : Currency}
      \end{abstractclass}
    \end{tikzpicture}
  \end{center}

  \bigskip{A operação {\tt calcularSaldo()} não possui implementação, serve como contrato de 
    nomes de operações a serem implementadas.}

\end{frame}


\begin{frame}{Objeto}

  \begin{itemize}
  \item Instância de uma classe.
  \end{itemize}

  \begin{center}
    \begin{tikzpicture}
      \begin{object}[text width=7cm]{conta da Alice : ContaBancaria}{0,0}
        \attribute{numero = 12345}
        \attribute{correntista = Alice}
        \attribute{saldo = 100}
    
        \operation{depositar(valor : Currency)}
        \operation{\it calcularSaldo() : Currency}
      \end{object}
    \end{tikzpicture}
  \end{center}
\end{frame}

\transitionslide{Relacionamentos}

\begin{frame}{Generalização}
  
  \begin{tikzpicture}

    \begin{class}{Shape}{1,0}
      \attribute{origin: Point}
      \operation{move(p : Point)}
      \operation{display()}
    \end{class}

    \begin{class}[text width=2.5cm]{Rectangle}{-4.5,-3.5}
      \inherit{Shape}
      \attribute{corner: Point}
    \end{class}

    \begin{class}[text width=2.5cm]{Circle}{-1,-3.5}
      \inherit{Shape}
      \attribute{radius: Float}
    \end{class}

    \begin{class}[text width=2.5cm]{Polygon}{3,-3}
      \inherit{Shape}
      \attribute{points: List}
      \operation{display()}
    \end{class}


  \end{tikzpicture}

\end{frame}


\begin{frame}{Dependência}\small

  \begin{itemize}
  \item Um relacionamento de \alert{dependência} indica que uma classe utiliza a outra.
  \end{itemize}


  \begin{tikzpicture}

    \begin{class}[text width=3cm]{Window}{-3,0}
      \operation{open()}
      \operation{close()}
      \operation{move()}
      \operation{display()}
      \operation{handleEvent()}
    \end{class}

    \begin{class}[text width=3cm]{Event}{3,0}
    \end{class}

    \draw[umlcd style dashed line ,->] (Window.east) |- node[black]{} (Event);
  
  \end{tikzpicture}

  \bigskip {Uma alteração na classe {\tt Event} afeta {\tt Window},  mas não o contrário.}

\end{frame}

\begin{frame}{Associação}\small

  \begin{itemize}
  \item Uma \alert{associação} é um relacionamento estrutural que especifica que as classes
    estão conectadas umas as outras. Junto à ligação pode ser indicado os \alert{papéis} e a
    \alert{multiplicidade} entre as classes.
  \end{itemize}\bigskip


  \begin{tikzpicture}

    \begin{class}[text width=3cm]{Pessoa}{-4,0}
    \end{class}

    \begin{class}[text width=3cm]{Empresa}{4,0}
    \end{class}

    \association{Pessoa}{empregado}{1..*}{Empresa}{empregador}{*}

  \end{tikzpicture}

\end{frame}


\begin{frame}{Agregação}
  
  \begin{tikzpicture}

    \begin{class}[text width=3cm]{Carro}{-3,0}
    \end{class}

    \begin{class}[text width=3cm]{Roda}{3,0}
    \end{class}

    \aggregation{Carro}{rodas}{4}{Roda}

  \end{tikzpicture}

\end{frame}


\begin{frame}[fragile]{Pacote}\footnotesize

  \begin{itemize}
  \item Reune interfaces e classes que possuem características comuns.
  \end{itemize}

  \begin{center}
    \begin{tikzpicture}

      \begin{package}{Contas}

        \begin{abstractclass}{ContaBancaria}{-1,0}
          \attribute{numero : Integer}
          \attribute{correntista : String}
          \attribute{saldo : Currency = 0}
    
          \operation{depositar(valor : Currency)}
          \operation{\it calcularSaldo() : Currency}
        \end{abstractclass}

        \begin{class}[text width=5.5cm]{Poupanca}{2.25,-3.5}
          \inherit{ContaBancaria}
          \attribute{rendimento : Percentage}
    
          \operation{depositarMensalmente(valor : Currency)}
          \operation{calcularSaldo() : Currency}
        \end{class}

        \begin{class}[text width=5cm]{ContaCorrente}{-3.35,-3.75}
          \inherit{ContaBancaria}
          \attribute{limiteChequeEspecial : Currency}
    
          \operation{compensarCheque(valor : Currency)}
          \operation{calcularSaldo() : Currency}
        \end{class}


      \end{package}
    \end{tikzpicture}
  \end{center}
\end{frame}


\begin{frame}{Visibilidade}\small

  \begin{center}
    \begin{tikzpicture}

      \begin{class}{Classe}{0,0}
        \attribute{$+$ público}
        \attribute{$-$ privado}
        \attribute{$\#$ protegido}
        \attribute{$\sim$ pacote}
      \end{class}

      \onslide<2>{
        \begin{abstractclass}{ContaBancaria}{0,-4}
          \attribute{$\#$ numero : Integer}
          \attribute{$+$ correntista : String}
          \attribute{$-$ saldo : Currency = 0}
    
          \operation{$+$ depositar(valor : Currency)}
          \operation{$\#$ \it calcularSaldo() : Currency}
        \end{abstractclass}
      }

    \end{tikzpicture}
  \end{center}

\end{frame}

\begin{frame}[fragile]{Referências}

  \begin{thebibliography}{Rumbaugh, 1998}

  \bibitem[Rumbaugh, 1998]{rumbaugh1998}
    James Rumbaugh, Ivar Jacobson, Grady Booch.
    \newblock 	   {\em The Unified Modeling Language User Guide}.
    \newblock	   Addison-Wesley, 1998.
 	   
  \bibitem[Blaha, 2005]{blaha2005}
    Michael Blaha, James Rumbaugh.
    \newblock	  {\em Modelagem e Projetos Baseados em Objetos com UML 2}.
    \newblock	  2$^a$ edição, Editora Campus, 2005.

  \bibitem[]{Yuan, 2012} Yuan Xu.
    \newblock {\em Drawing UML Class Diagram by using} {\tt pgf-umlcd}.
    \newblock \url{http://goo.gl/x1W2LV}

  \end{thebibliography}

\end{frame}

