% Fonte: http://users.csc.calpoly.edu/~jdalbey/205/Lectures/myths.html

\lecture{Mitos de Desenvolvimento de Software}{myths}

\lecturetitle{\course}{\insertlecture}

\section*{\insertlecture}

\frame{\maketitle}

\begin{frame}{Mítico Homêm-Mês}
\footnotesize
  \begin{columns}
    \begin{column}{.4\textwidth}
      \includegraphics[scale=.35]{img/mitico.png}
    \end{column}
    \begin{column}{.6\textwidth}
      \begin{itemize}
      \item Observações do autor sobre a experiência de desenvolvimento 
        do sistema operacional OS/360 (1964--*) na IBM.
      \item 1$^a$ publicação: 1975.
      \end{itemize}
    \end{column}

  \end{columns}
  
\end{frame}

\begin{frame}{\inserttitle}

Os mitos são classificados de acordo com o ponto de vista de quem 
os emprega em mitos de:

  \begin{enumerate}[<+-| alert@+>]
  \item Gerenciamento;
  \item Desenvolvimento;
  \item Desenvolvedor.
  \end{enumerate}

\end{frame}

\note{A seguir são descritos alguns mitos que normalmente conduzem à
produção de software com defeitos, atraso na entrega ou cancelamento
do projeto. Os mitos foram classificados de acordo com o ponto de
vista de quem os emprega.}

\begin{frame}{Mitos de Gerenciamento}
\small
\begin{description}[<+-| alert@+>]
\item[Normas e padrões] Normas são editadas por empresas e comitês, às
  vezes são úteis. Porém, podem se tornar irrelevantes, incompletas e
  incompreensíveis se não estiverem vinculadas à realidade.
\item[Ferramentas] Ferramentas podem ajudar, porém não fazem mágica. A
  solução de problemas requerem mais que ferramentas, requerem um
  profundo entendimento do problema e da solução. Segundo Fred
  Brooks~\cite{brooks1975}, não há ``bala de prata'' no
  desenvolvimento de software.
\item[Mais programadores:] A solução intuitiva de que adicionando mais
  programadores pode fazer com que um projeto cumpra seu prazo é
  completamente errônea. Este procedimento causa sobrecarga de
  comunicação e atraso devido ao tempo que os novos integrantes da
  equipe leva para entender o problemas e solução. Segundo Fred
  Brooks~\cite{brooks1975}, ``adicionando pessoas a um projeto
  atrasado torna-o mais atrasado''.
\end{description}

\end{frame}

\begin{frame}{Mitos de Desenvolvimento}

\begin{description}[<+-| alert@+>]
\item[Alterações são fáceis:] Alterações em um software não são
  facilmente acomodadas como a princípio parecem ser. As alterações
  podem tornar o software mais complexo, introduzir novos erros e
  exigir uma quantidade enorme de trabalho devido às modificações em
  todos os processos relacionados (teste, documentação, $\ldots$).

\item[Visão superficial:] {\bf Uma visão geral da solução é suficiente
    para começar a programar.} É necessário um visão concreta sobre os
  requisitos, para detalhar os problemas e soluções para dar início ao
  desenvolvimento.
\end{description}

\end{frame}

\begin{frame}{Mitos do Desenvolvedor}

\begin{itemize}[<+-| alert@+>]
\item {\bf O trabalho acaba quando o programa é entregue:} O software requer
  atenção após a entrega, a manutenção, aperfeiçoamento e extensões
  geram uma necessidade contínua de suporte.
\item {\bf O sucesso de um projeto depende somente da qualidade do software
  entregue:} Documentação e informações de configuração são
  extremamente importantes.
\item {\bf Não há como avaliar a qualidade de um software até que ele
  esteja executando:} Devido a natureza abstrata do software, ele pode
  ser avaliado sem que uma linha de código seja produzida. Há métodos
  formais de análise para verificação de pontos críticos com relação à
  corretude, segurança e confiabilidade.
\end{itemize}

\end{frame}

\begin{frame}{Referências}
\bibliographystyle{plain}
\bibliography{software-engineering}
\end{frame}


%%
% Local variables:
% mode: latex
% mode:auto-fill
% TeX-file: main
% End:
%%
