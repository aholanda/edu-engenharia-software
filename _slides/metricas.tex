\lecture{Métricas}{metrics}

\lecturetitle{\insertlecture}{\course}

\frame{\maketitle}

\section{\insertlecture}


\begin{frame}{Algumas Métricas}
  \scriptsize
  Medidas que podem ser usadas para monitoramento e controle 
  do processo de desenvolvimento.

  \begin{itemize}[<+-| alert@+>]\setbeamercovered{transparent}
  \item {\bf Número de linhas de código}
  \item {\bf Extensão ({\it coverage}) do código}
  \item {\bf Análise de pontos por função}
  \item {\bf Densidade de defeitos do módulo} {\it bugs\/} pelo tamanho do módulo.
  \item {\bf Refugo do sistema} tempo dispendido no reparo de falhas pelo tempo 
    total de desenvolvimento.
  \item {\bf Estabilidade dos requisitos} número de requisitos no levantamento pelo 
    número de requisitos na fase final do projeto.
  \item {\bf Defeitos no teste} número de defeitos durante a fase de testes.
  \item $\ldots$
  \end{itemize}
\end{frame}

\subsection{Linhas de Código--SLOC}

\begin{frame}{Número de Linhas de Código}{SLOC}
\begin{itemize}[<+-| alert@+>]\setbeamercovered{transparent}
\item O {\bf número de linhas de código} (SLOC--{\em Source Lines of
    Code}) é uma métrica usada para saber o tamanho do software pela
  contagem do número de linhas de código fonte.
\item Tenta quantificar o esforço de produção, produtividade e
  manutenibilidade.
\item Umas das métricas mais antigas.
\end{itemize}
\end{frame}

\begin{frame}{Exemplo SLOC}\scriptsize

\begin{center}
\begin{tabular}{l|l|l}
\bf\hfil Ano &\bf\hfil	Sistema &\bf\hfil SLOC (milhões) \\\hline
1993&	Windows NT 3.1&	4--5\\
1994&	Windows NT 3.5&	7--8\\
1996&	Windows NT 4.0&	11--12\\
2000&	Windows 2000&	> 29\\
2001&	Windows XP&	45\\
2003&	Windows Server 2003 &	50\\\hline
\end{tabular}\\
\end{center}
{\scriptsize Fonte: \href{http://www.knowing.net/index.php/2005/12/06/how-many-lines-of-code-in-windows/}{``How Many Lines of Code in Windows?''}. Knowing .NET. }

\pause\bigskip
\begin{center}
\begin{tabular}{l|l|l}\hline
\bf\hfil Ano & \bf\hfil Sistema & SLOC (milhões) \\\hline
  2001&	Linux kernel 2.4.2&	2,4\\
  2003&	Linux kernel 2.6.0&	5,2\\
  2009&	Linux kernel 2.6.29&	11,0\\
  2009&	Linux kernel 2.6.32&	12,6\\
  2010&	Linux kernel 2.6.35&	13,5\\
  2012&	Linux kernel 3.6&	15,9\\
  2015-06-30&	Linux kernel pre-4.2&	20,2\\\hline
\end{tabular}\\
\end{center}
{\scriptsize Fonte: \href{https://en.wikipedia.org/wiki/Source_lines_of_code}{Wikipedia}}
\end{frame}

\begin{frame}[fragile]{SLOC}{Problemas}\scriptsize
  \lstset{basicstyle=\scriptsize,numberstyle=\tiny,frame=single}
  Cada linguagem possui uma expressividade diferente. Exemplo: 
  comparação do algoritmo {\em quicksort} em C e Haskell.
  \begin{center}
      \begin{lstlisting}[language=C]
        void quicksort (int *a, int n) {
          int i, j, p, t;
          if (n < 2)  return;
          p = a[n / 2];
          for (i = 0, j = n - 1;; i++, j--) {
            while (a[i] < p) i++;
            while (p < a[j]) j--;
            if (i >= j) break;
            t = a[i], a[i] = a[j], a[j] = t;
          }
          quicksort(a, i);
          quicksort(a + i, n - i);
        }
      \end{lstlisting}\pause

      \begin{lstlisting}[language=haskell]
        import Data.List
 
        qsort [] = []
        qsort (x:xs) = qsort ys ++ x : qsort zs where (ys, zs) = partition (< x) xs
      \end{lstlisting}
    \end{center}
  {\tiny Fonte: \href{https://rosettacode.org/wiki/Sorting_algorithms/Quicksort}{Rosetta Code}.}
\end{frame}


\begin{frame}[fragile]{SLOC}{Problemas}

  Diferentes estilos de programação levam a diferentes resultados.

  \lstset{basicstyle=\scriptsize,numberstyle=\tiny,frame=single}
\begin{center}
  \begin{lstlisting}[language=C]
    for (i = 0, j = n - 1;; i++, j--) {
      while (a[i] < p) i++;
      while (p < a[j]) j--;
      if (i >= j) break;
      t = a[i], a[i] = a[j], a[j] = t;
    }
  \end{lstlisting}
  
  \begin{lstlisting}[language=C]
    for (i = 0, j = n - 1;; i++, j--) 
    {
      while (a[i] < p) 
         i++;
      while (p < a[j]) 
          j--;
      if (i >= j) 
          break;
      t = a[i]; 
      a[i] = a[j]; 
      a[j] = t;
    }
  \end{lstlisting}
\end{center}
\end{frame}

\begin{frame}{SLOC}{Vantagens}
  
  \begin{itemize}[<+-| alert@+>]\setbeamercovered{transparent}
  \item Facilidade de automação;
  \item Métrica intuitiva;
  \item Facilidade de entendimento.
  \end{itemize}
\end{frame}

\begin{frame}{SLOC}{Desvantagens}
    \begin{itemize}[<+-| alert@+>]\setbeamercovered{transparent}
  \item Falta de ligação com a funcionalidade;
  \item Falta de normas para contagem;
  \item Incentiva o aumento do código ao invés da melhoria;
  \item Não leva em conta: linguagem de programação, experiência do programador, 
    convenção de codificação.
  \end{itemize}
\end{frame}

\begin{frame}{Extensão ({\em coverage} do código)}\small
  
  Métrica de verficação da porcentagem de código executado em uma
  suíte de testes, e pode ser classificada de acordo com o critério de
  medição:

  \begin{description}[<+-| alert@+>]\setbeamercovered{transparent}
  \item[Função:] calcula a porcentagem de funções chamadas;
  \item[Expressão:] calcula a porcentagem de expressões executadas;
  \item[Desvios de fluxo:] calcula a porcentagem e verifica o caminho
    feito pelas estruturas de controle;
  \item[Condicional:] caminho coberto pelo fluxo de execução em
    estruturas booleanas.
  \end{description}

  \pause\bigskip

  O objetivo é quantificar a porcentagem de código não coberta pelos testes, 
  em que pode haver existência de {\em bugs} que não serão avaliados.
\end{frame}
