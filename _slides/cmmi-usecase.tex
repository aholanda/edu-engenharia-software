\documentclass[12pt]{../notes}


\title{Engenharia de Software II}
\author{Adriano J. Holanda}
\date{21/8/2018}

\begin{document}
\maketitle

\section*{CMMI: Caso de Uso}

Responda às questões a seguir, baseando-se (somente) no artigo
\href{https://msdn.microsoft.com/pt-br/library/cc517970.aspx}{Expandindo
  o Agile para se adequar ao CMMI Nível 3} de David J. Anderson da
Empresa Microsoft.

\begin{enumerate}
\item Qual o mestre da qualidade cujas ideias foram adotadas na construção
  de uma metodologia e quais são essas ideias?

\item Por que era assumida a postura de a mistura Ágil e CMMI seriam como
  água e óleo?

\item Por que o pensamento de Deming está muito mais alinhado ao manifesto ágil?

\item O CMM de Software (precursor do CMMI) foi fortemente influenciado
  por outro mestre da qualidade. Quem era este mestre, e qual princípio
  destacado no texto associado a este mestre?

\item Trecho extraído do texto: ``Se devemos seguir uma abordagem de
  Deming para atingir o nível 5 do CMMI, deve haver um mecanismo que
  permita a monitoria do controle estatístico. A compatibilidade com
  métodos ágeis pode ser alcançada pela mensuração da velocidade de
  produção dos itens valorizados pelo cliente e pela utilização dessa
  métrica no planejamento.'' \\Qual são os mecanismos adotados para
  realizar a ``monitoria do controle estatístico'' citados no texto?

\item Qual a essência do planejamento iterativo e adaptável?
  
\item Como é possível chegar a um método aǵil comparável ao CMMI nível~5?
  
\item Quais são as principais ideias do manifesto ágil?

\item Quais principais as diferenças entre metodologia de
  desenvolvimento de software (Ágil) e modelo de processos (CMMI)?

  
\end{enumerate}

\end{document}
