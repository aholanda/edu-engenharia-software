%% FUTURE: http://ix.cs.uoregon.edu/~michal/book/slides/pdf/PezzeYoung-Ch05-FiniteModels.pdf

\lecture{Desenvolvimento baseado em testes}{tests}
\lecturetitle{\insertlecture}{\course}
\maketitle

\begin{frame}{Testes}{TDD}\small
  Os testes são importantes nas abordagens Ágeis que seguem o
  desenvolvimento baseado em testes (Test-Driven Development--TDD)
  e segue os seguintes passos:
  \begin{itemize}[<+-| alert@+>]\setbeamercovered{transparent}
  \item A partir das histórias do usuário elaborar testes de unidade
    que falham para testar o código não existente;
  \item Escreva somente o código suficiente para passar no teste
    e procure oportunidades para refatorar ({\em refactoring}), esta
    sequência é chamada vermelho--verde--refatoramento;
  \item Use {\em mocks} e {\em stubs} nos testes para isolar o
    comportamento do código testado do comportamento de outras classes
    ou métodos que dependem deles;
  \item Vários testes de cobertura de código podem ser usados para
    verificar que partes do código ainda precisam ser testadas.
  \item Os testes param quando um nível de cobertura á alcançado, por
    exemplo, 95\%.
  \end{itemize}  
\end{frame}

\begin{frame}{Testes}{TDD}\small

  \begin{quote}
    ``O teste do programa pode ser uma maneira muito efetiva de mostrar
    a presença de {\em bugs}, mas irremediavelmente inadequado para
    mostrar a ausências destes.'' Edsger W. Dijkstra
  \end{quote}
  
  \begin{itemize}[<+-| alert@+>]\setbeamercovered{transparent}
  \item As metologias tradicionais também possuem os mesmos conceitos,
    porém, em uma ordem diferente e executado por pessoas diferentes.
  \item Uma alternativa aos testes é o uso de métodos formais. Eles
    usam especificações formais do comportamento correto do programa
    que são verificadas por provadores automáticos de teoremas ou
    por buscas exaustivas dos estados possíveis, ambos vão além do que
    os testes podem fazer.
  \end{itemize}
  
\end{frame}

\begin{frame}{Propriedades dos Testes}

  As propriedades desejável do teste são:
  
  \begin{description}[<+-| alert@+>]\setbeamercovered{transparent}
  \item[Rapidez:] ele deve ser simples e rápido para não interferir
    no fluxo de raciocínio;
  \item[Independente:] Nenhum teste deve depender em condições
    criadas por outros testes;
  \item[Reprodutível:] o teste não deve depender de fatores externos
    como a data do dia ou ``constantes mágicas'';
  \item[Automático:] o teste deve determinar por si próprio se passou
    ou falhou sem interferência humana;
  \item[Atualizado:] O código deve ser criado e atualizado ao mesmo
    tempo que o código que está testando.
  \end{description}
  
\end{frame}

\begin{frame}{Testes de Unidade}

\begin{itemize}[<+-| alert@+>]
\item Testes são uma forma antiga de assegurar que algumas alterações 
no código não tenham modificado a saída esperada para determinadas
funcionalidades.
\item O {\bf Teste de Unidade} ({\em Unit Testing}) foi definido para 
ser parte integrante da metodologia ágil {\em Extreme Programming}
(XP), com o objetivo de verificar se as unidades selecionadas para
teste, sejam elas módulos, classes ou métodos, retornam os resultados
esperados após alteração no código. Caso contrário, a construção do
software irá falhar, indicando que houve inserção de {\em bug}.
\end{itemize}

\end{frame}

\lstset{language=java,basicstyle=\scriptsize}
\begin{frame}[fragile]{Exemplo de Teste usando JUnit}

\lstinputlisting{Calculadora.java}

\tiny
Para compilar: {\tt \$ javac Calculadora.java}

\end{frame}

\begin{frame}[fragile]{Exemplo de Teste usando JUnit, cont.}

\lstinputlisting{TesteCalculadora.java}

\tiny
Para compilar:{\tt \$ javac -cp .:junit-4.12.jar TesteCalculadora.java Calculadora.java}\\
Para executar: {\tt \$ java -cp .:junit-4.12.jar:hamcrest-core-1.3.jar org.junit.runner.JUnitCore TesteCalculadora}
\end{frame}


\begin{frame}{Exercício}

  Escreva testes para os seguintes métodos:
  \begin{description}
  \item[mul(int x, int y)] - retorna o valor da multiplicação entre {\tt x} e {\tt y}
  \item[dif(int x, int y)] - retorna o valor da diferença entre {\tt x} e {\tt y};  
  \item[max(int x, int y)] - retorna o maior valor entre {\tt x} e {\tt y};
  \item[min(int x, int y)] - retorna o menor valor entre {\tt x} e {\tt y};
  \item[media(List xs)] - retorna a média do valores na lista {\tt xs}.
  \end{description}
  
\end{frame}

\begin{frame}{Referência}
  \begin{thebibliography}{10}
    \beamertemplatebookbibitems
  \bibitem[Fox and Patterson, 2015]{fox2015}
    Armando Fox and David Patterson.
    \newblock ``Construindo Software como Serviço (SaaS): Uma Abordagem Ágil Usando Computação em Nuvem ''.
  \newblock Strawberry Canyon LLC, 2015.
  \end{thebibliography}
\end{frame}