\lecturetitle{Sistema de Controle de Versão}{\course}

\frame{\maketitle}

\begin{frame}{Sistema de Controle de Versão}{Vantagens}
\begin{itemize}[<+-| alert@+>]
\item Permite mais de um desenvolvedor trabalhar no mesmo código;
\item Versões mais antigas são facilmente recuperáveis;
\item Mantém registro das modificações;
\item Permite comparação das diferenças entre as versões;
\item Permite ramificações ({\em branches}) e versões múltiplas 
      do mesmo projeto.
\end{itemize}
\end{frame}

\tikzset{object/.style={rounded rectangle,draw}
         ,version/.style={object,minimum width=2.5cm,fill=green!77!black}
         ,file/.style={object,minimum width=2cm,fill=blue!30}
         ,scmNode/.style={fill=gray!50,draw}
         ,computer/.style={scmNode,minimum width=3.5cm,minimum height=2cm}
         ,server/.style={scmNode,minimum width=3.5cm,minimum height=5cm}
}      
\newcommand\versionDatabase[1]{
        \node[minimum height=3.25cm,minimum width=2.5cm,fill=white,draw] (#1VersionDatabase) at (#1) {};
        \node[version] (#1Version2) at (#1) {Versão 2};
        \node[version] (#1Version1) [below of=#1Version2] {Versão 1};
        \node[version] (#1Version3) [above of=#1Version2] {Versão 3};
        \path[draw] (#1Version3) -- (#1Version2);
        \path[draw] (#1Version2) -- (#1Version1);
}
\begin{frame}{Sistema de Controle de Versão Centralizado}

Exemplos: \href{https://subversion.apache.org/}{Subversion}, 
 \href{http://www.nongnu.org/cvs/}{CVS}, 
\href{http://bazaar.canonical.com/en/}{Bazaar}, 

\begin{center}
\begin{tikzpicture}
\node[server] (Server) {};
\node[below]  (ServerLabel) at (Server.north) {Servidor};

\versionDatabase{Server}

\node[computer] [left of=Server,xshift=-3cm,yshift=1.15cm] (ComputerA) {};
\node[below] at (ComputerA.north) {Computador A};
\node[file] (FileA) at (ComputerA) {Arquivo};
\node[computer] (ComputerB) [below of=ComputerA,yshift=-1.5cm] {};  
\node[below] at (ComputerB.north) {Computador B};
\node[file] (FileB) at (ComputerB) {Arquivo};
\path[->,draw] (ServerVersionDatabase) -> (FileA);
\path[->,draw] (ServerVersionDatabase) -> (FileB);
\end{tikzpicture}
\end{center}
\tiny Fonte: Vídeo do youtube: \href{https://www.youtube.com/watch?v=ZDR433b0HJY}{``Introduction to Git with Scott Chacon of GitHub.''}

\end{frame}

\begin{frame}{Sistema de Controle de Versão Distribuído}
Exemplos: \href{http://gitscm.org/}{Git},
          \href{https://www.mercurial-scm.org/}{Mercurial},
          
\begin{center}
\begin{tikzpicture}[distributed/.style={minimum height=5cm}]

\node[server] (Server) {};
\node[below]  (ServerLabel) at (Server.north) {Servidor};
\versionDatabase{Server}

\node[computer,distributed] [left of=Server,xshift=-3cm,yshift=-.5cm] (ComputerA) {};
\node[below] at (ComputerA.north) {Computador A};
\versionDatabase{ComputerA}
\node[file] (FileA) [below of=ComputerAVersion1] {Arquivo};

\node[computer,distributed] (ComputerB) [right of=Server,xshift=3cm,yshift=-0.5cm] {};  
\node[below] at (ComputerB.north) {Computador B};
\versionDatabase{ComputerB}
\node[file] (FileB)  [below of=ComputerBVersion1] {Arquivo};

\path[>=latex,<->,draw] (Server) -> (ComputerA);
\path[>=latex,<->,draw] (Server) -> (ComputerB);
\end{tikzpicture}
\end{center}
\tiny Fonte: Vídeo do youtube: \href{https://www.youtube.com/watch?v=ZDR433b0HJY}{``Introduction to Git with Scott Chacon of GitHub.''}
\end{frame}

\begin{frame}{Sistema de Controle de Versão Distribuído}{Vantagens}
\begin{itemize}[<+-| alert@+>]
\item Cada desenvolvedor pode ter sua cópia do projeto de software.
\item Não sobrecarrega a rede a cada alteração e consolidação do 
código modificado.
\item As ramificações ({\em branches}) são locais e podem ser 
 fundidas com o repositório central.
\item Os desenvolvedores podem compartilhar código entre si 
antes de transferir para o repositório central.
\end{itemize}
\end{frame}
