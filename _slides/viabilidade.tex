\lecture{Análise de Viabilidade}

\lecturetitle{\course}{\insertlecture}

\frame{\maketitle}

\begin{frame}{\insertlecture}
  
  Após o levantamento de requisitos, a \alert{\insertlecture\ } deve ser
  feita para definir se o projeto do sistema será aceito para
  desenvolvimento.\bigskip\pause
  
  A \insertlecture\ é difícil de ser realizadas devido às incertezas geradas por:
  \begin{itemize}[<+->]\setbeamercovered{transparent}
  \item incertezas do escopo do projeto, os \alert{clientes} normalmente têm 
    dificuldade em definir com precisão todos os requisitos;
  \item dificuldades na definição dos \alert{recursos} a serem utilizados e
    \alert{cronograma} de execução do projeto;
  \item necessidade de \alert{mudanças} na organização.
  \end{itemize}

  \bigskip\pause Soluções \alert{alternativas} para o problema devem
  ser avaliadas, bem como, a possibilidade de terceirização de alguma
  etapa do processo.

\end{frame}

\begin{frame}{Classificação da Análise}\small
  \begin{description}[<+->]\setbeamercovered{transparent}
  \item[Viabilidade técnica:] confronta as opções de tecnologias 
    a serem utilizadas para o desenvolvimento do sistema com o 
    domínio da equipe destas tecnologias.
  \item[Viabilidade operacional:] determina o grau em que o projeto proposto 
    se encaixa com:
    \begin{itemize}[<+->]\setbeamercovered{transparent}
      \item a cultura da organização;
      \item os processos existentes;
      \item cronograma de projetos em andamento.
      \end{itemize}
    \item[Viabilidade econômica:] determina o efeito da produção do sistema, 
    segundo o projeto proposto, nas finanças da organização.
  \item[Viabilidade legal:] determina se o sistema proposto está em conformidade 
    com a legislação vigente no país.
  \end{description}
\end{frame}

\begin{frame}{Identificação dos Riscos}\small
  
  A \insertlecture\ visa identificar os \alert{riscos} técnicos, operacionais, 
  econômicos e legais caso a organização aceite o projeto.

  \bigskip\pause

  Ela ajuda a elaborar um plano de contingência para cada risco identificado. Por exemplo:
  \begin{itemize}[<+->]\setbeamercovered{transparent}
  \item O que fazer se o uso da linguagem X com o banco de dados BZ se mostrar inviável?
  \item O que fazer se o programador Beto se afastar por motivo de saúde por X dias?
  \item O que fazer se  houver problemas com o servidor de desenvolvimento e o deixar 
    indisponível por X horas?
  \item O que fazer se o preço da licença do software WW usado para o desenvolvimento 
    {\em web} aumentar de valor ao ser renovada?
  \end{itemize}

  \pause
  
  Quanto mais dificuldades forem analisadas nesta fase, menor a
  probabilidade da organização sofrer grandes impactos no cronograma
  de execução e orçamento do projeto.
\end{frame}

\begin{frame}{\insertlecture}{Escopo}
  
  O \alert{escopo} expressa as \alert{fronteiras} (requisitos) do sistema e deve conter:
  \begin{itemize}[<+->]\setbeamercovered{transparent}
  \item lista de \alert{funções incluídas};
  \item lista de \alert{funções excluídas};
  \item lista de \alert{dependências};
  \item lista dos \alert{sistemas atuais a serem substituídos}.
  \end{itemize}

  \bigskip\pause
  Confusão sobre o escopo costuma ser a principal razão da insatisfação 
  do cliente com o sistema.
\end{frame}

\begin{frame}{\insertlecture}{Benefícios}
  
  Benefícios da \insertlecture:

  \begin{itemize}[<+->]\setbeamercovered{transparent}
  \item Criação de um produto ``vendável'';
  \item Melhora a eficiência da organização;
  \item Organiza um processo de produção complexo;
  \item Possibilidade de inovação ou melhoria dos processo existentes;
  \item Proteção e segurança.
  \end{itemize}
\end{frame}

\begin{frame}{\insertlecture}{Técnica}
  
  A \insertlecture\ precisa demonstrar que o sistema proposto é 
  tecnicamente viável. Isto requer:
  
  \begin{itemize}[<+->]\setbeamercovered{transparent}
  \item um \alert{esboço} dos requisitos;
  \item um \alert{esboço} do modelo ou especificação;
  \item opções de módulos a serem adquiridos;
  \item \alert{estimativa} da quantidade de usuários, dados, transações e outros 
    elementos que o sistema tratará.
  \end{itemize}

  \pause\bigskip

  Estes números aproximados farão parte do plano de provisionamento de recursos: 
  equipe, cronograma, equipamentos.

\end{frame}

\begin{frame}{Técnicas para \insertlecture}
  O \alert{objetivo de maior prioridade} é assegurar que o cliente e a
  equipe de desenvolvimento tenham o mesmo entendimento sobre os
  requisitos do sistema.
  \pause\bigskip
  
  Para a \alert{equipe de desenvolvimento} entender os requisitos é necessário:
  \begin{itemize}[<+->]\setbeamercovered{transparent}
  \item Entrevista com o cliente e os funcionários da organização do cliente;
  \item Revisão dos sistemas já existentes (incluindo concorrentes).
  \end{itemize}
  
  \pause\bigskip
  Para o \alert{cliente} apreciar a proposta é necessário:
  \begin{itemize}[<+->]\setbeamercovered{transparent}
  \item Demonstração das principais funcionalidade de sistemas similares;
  \item Demonstração da imagem da interface com o usuário;
  \item Rever os principais processos a serem implementados.
  \end{itemize}
\end{frame}

\begin{frame}{Técnicas para \insertlecture}
  \alert{Esboço do orçamento:}
  \begin{itemize}[<+->]\setbeamercovered{transparent}
  \item $n$ pessoas por $m$ meses a R\$~$X$ por mês;
  \item equipamentos, custos fixos, licenças de software;
  \item contingência.
  \end{itemize}
  
  \pause\bigskip
  
  \alert{Fases ou etapas de desenvolvimento}
  \begin{itemize}
  \item especificar datas aproximadas de entregas dos produtos gerados em cada etapa;
  \item estimar versões a serem lançadas.
  \end{itemize}  
\end{frame}

\begin{frame}{Relatório de Visibilidade}
  
  A \insertlecture\ pode gerar um \alert{relatório de viabilidade} com 
as seguintes características:

\begin{itemize}
\item Escrito para uma audiência geral: cliente, administradores, desenvolvedores, por exemplo;
\item Curto o suficiente para que todos possam ler;
\item Longo o suficiente para que nenhum detalhe importante seja omitido;
\item Detalhes técnicos ou adicionais podem ser anexados ao final do documento.
\end{itemize}

\pause\bigskip

Um relatório que não é facilmente entendido é possui pouca utilidade.

\end{frame}

\begin{frame}{Referência}
  \begin{itemize}
  \item William Y.\ Arms. CS 5150: Software Engineering, Fall 2014. \href{http://www.cs.cornell.edu/courses/cs5150/2014fa/slides/C1-feasibility.pdf}{Feasibility studies}.
  \item \href{https://goo.gl/tpVxoD}{Feasibility study}. Wikipedia.
  \end{itemize}
\end{frame}