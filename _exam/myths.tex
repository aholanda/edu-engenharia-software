\question{0,5}~Assinale a alternativa NÃO é mito de desenvolvimento de
software:

\begin{enumerate}[a)]
\item O trabalho é terminado quando o sistema é entregue.
\item O sucesso do projeto do software depende somente do produto
  entregue e não da documentação.
\item Quando o projeto estiver atrasado, basta adicionar mais programadores.
\item Para resolver problemas de desenvolvimento não basta usar ferramentas.
\item Problemas de desenvolvimento podem ser solucionados pelo uso de normas e padrões.
\end{enumerate}

\question{1,0} Por quê a simples utilização de padrões nos processos
de desenvolvimento não assegura um produto (software) correto?

\question{2,0} Mito de gerenciamento: Por quê ``adicionando pessoas a
um projeto [de software] atrasado torna-o mais atrasado''?

\question{4} Leia atentamente o texto a seguir:

\begin{quote}\small
  ``O projeto Portal do Paciente do Hospital Dante Divino contou com
  especificações da equipe interna de analistas de negócios da
  empresa, que aplicou em seu trabalho conceitos baseados nas boas
  práticas do guia BABOK ({\em Business Analysis Body of Knowledge}),
  além de uma metodologia própria de especificação que tem como
  referência os modelos do RUP ({\em Rational Unified Process}). O
  desenvolvimento das funcionalidades foi realizado por uma equipe
  própria de desenvolvedores que utilizou a plataforma Microsoft .NET
  como base para o projeto. O tempo aproximado de especificação e
  desenvolvimento do atual portal foi de um ano e dez meses. Durante
  este período, foi adotada a estratégia de entregas periódicas dos
  pacotes com novidades do sistema, deixando explícito ao cliente a
  intenção da instituição de melhorar constantemente a qualidade dos
  serviços prestados.''
\end{quote}

\begin{enumerate}[a)]

  \item Os padrões citados no texto podem colaborar para o desenvolvimento do sistemas ou
  constituem um mito? Justifique sua resposta.
 % descrevendo a influência
 % da adoção de padrões no processo de produção de software.

  \item Qual o processo de desenvolvimento que mais se adapta ao sistema
  a ser desenvolvido? Justifique sua resposta.

\end{enumerate}
