\section*{Projeto Orientado a Objetos e Modularização}

\Q Um disco rígido é composto por cabeçote (2), 4 bandejas (4), eixo
(2), carcaça (1) e motor. O motor é composto por solenoide (3), 4
cabos (4), base (1), rolamento (2) e eixo do motor (2).  Faça um
diagrama de classes representando estas relações. Projete um método
nas classes que calcule o custo de cada objeto a partir dos
componentes, se houver. O custo de cada componente está especificado
entre parênteses. Projete o método que calcula o custo de forma que
tenha a mesma assinatura para todas as classes.

\Q Um software de áudio digital utiliza os conceitos de {\tt
  Instrumento}, {\tt Bateria}, {\tt InstrumentoCorda}, {\tt Viola},
{\tt Violão}, {\tt Guitarra}, {\tt InstrumentoSopro}, {\tt Saxofone},
{\tt Flauta} e {\tt Piano}. Projete, de forma que seja
possível o acoplamento dinâmico, o diagrama de classes destes
conceitos a serem implementados no software. Todas as classes devem
conter os métodos {\tt tocar} e {\tt afinar}. Faça a sobrecarga do
método {\tt afinar} que receba como parâmetro o tom em que o
instrumento de corda deve ser afinado.

\Q Faça um protótipo do diagrama do exercício anterior em 3 linguagens
de programação com suporte a orientação a objetos. Quando os métodos
forem invocados, faça-os imprimir o nome da classe e assinatura. Por
exemplo, para o método {\tt tocar} da classe {\tt Bateria}, a
implementação em Java será:

\medskip\hfil{\small\tt public void tocar() \{ System.out.println("Bateria.tocar()");\}}

\Q Defina API (Interface de Programação Pública), estabeleça a relação
com o projeto orientado a objetos e cite exemplos de sua utilização.

\Q Enumere os princípios de projeto modular estabelecendo uma analogia
e citando exemplos (sistemas operacionais, bibliotecas, linguagens de
programação).

\question{6} Desenhe os seguintes diagramas de classe UML:

\begin{enumerate}[a)]
\item As classes {\tt ContaBancaria}, {\tt Imovel} e {\tt Seguridade}
  implementam a interface {\tt Ativo}. As classes {\tt ContaCorrente}
  e {\tt Poupança} são especializações de {\tt ContaBancaria}. As
  classes {\tt Acoes} e {\tt Bonus} são especializações de {\tt
    Seguridade}.

\item Uma interface {\tt IURL} possui os métodos {\tt
    openConnection()}, {\tt parseURL()} e {\tt setURL}. A classe {\tt
    URL} implementa {\tt IURL}, e a classe {\tt Controller}
  especializa {\tt URL} sobrescrevendo o método {\tt openConnection()}
  e sobrecarregando o mesmo método que recebe um parâmetro do tipo
  {\tt Protocol}.

\item A classe {\tt DispositivoMovel} é composta das classes {\tt
    Processador}, {\tt Bateria}, {\tt Tela}, {\tt Memoria}, {\tt
    Audio}. Há também as classes {\tt EntradaAudio} e {\tt SaidaAudio}
  são especializações de {\tt Audio}.
\end{enumerate}


\question{3} Para o diagrama UML 

\begin{center}
\begin{tikzpicture}[font=\scriptsize]
  \begin{class}{Empresa}{0,0}
  \end{class}
  \begin{class}{Departamento}{-3,-2}
    \attribute{-nome: String}
  \end{class}
  \begin{class}{Escritorio}{3,-2}
    \attribute{-nomeFilial: String}
    \attribute{\#endereco: String}
    \attribute{+url: String}
  \end{class}

  \composition {Empresa}{}{}{Departamento}
  \composition {Empresa}{}{}{Escritorio}

  \begin{class}{EscritórioCentral}{3,-4}
    \inherit{Escritorio}
  \end{class}
  \association{Departamento}{}{}{Escritorio}{}{}

  \begin{class}{Pessoa}{-3,-5}
    \attribute{-nome: String}
    \attribute{-codigo: Integer}
    \attribute{-endereco: String}
    \operation{+obterNome(): String}
    \operation{+obterCodigo(): Integer}
    \operation{+obterContato(): Contato}
    \operation{$\overline{\hspace{4.5cm}}$}
  \end{class} 

  \association{Departamento.south east}{}{}{Pessoa.north east}{gerente}{1}
  \association{Departamento.south west}{}{}{Pessoa.north west}{membro}{*}

    \begin{class}{Contato}{4,-8}
      \attribute{-endereco: String}
      \attribute{-fone: Integer[\ ]}
    \end{class}
    
    \draw [umlcd style dashed line ,->] (Pessoa) --node [above, sloped, black]{} (Contato);

\end{tikzpicture}
\end{center}

\noindent responda às seguintes questões:

\begin{enumerate}[a)]
\item Descreva o tipo de relacionamento entre {\tt Empresa} e {\tt Departamento},
{\tt Empresa} e {\tt Escritório}, {\tt Departamento} e {\tt Pessoa}, {\tt Escritorio} 
e {\tt EscritorioCentral}, {\tt Pessoa} e {\tt Contato}.
\item Quais atributos {\tt EscritorioCentral} herda de {\tt Escritorio}?
\item Explique as cardinalidades e descreva os papéis de {\tt membro} e {\tt gerente}.
\item Adicione uma classe chamada {\tt Salario} que será usada pela classe {\tt Pessoa}. 
Adicione um método na classe {\tt Pessoa} para obter o salário.
\end{enumerate}

\question{1,0} Desenhe o diagrama de classes usando UML que represente
o código em Java listado a seguir:

\begin{center}
\begin{lstlisting}
interface Ator {
  void atuar();
}
interface Cantor {
  void cantar();
}
public class AtorMusical implements Ator, Cantor
{...}
\end{lstlisting}
\end{center}
