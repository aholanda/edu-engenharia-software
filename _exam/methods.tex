
\question{4,0}~Tendo as seguintes fases da Metodologia Cascata como base:

\begin{enumerate}
\item  Levantamento de Requisitos
\item Modelagem e Especificação
\item Implementação
\item Implantação e Validação
\end{enumerate}

Responda, de forma sucinta (mas não tanto), às seguintes questões:

\begin{enumerate}[a)]

\item [1 ponto(s)] Explique como a fase de levantamento de requisitos é
realizada e qual nome leva nas metodologias ágeis: Scrum, XP e
Kanban.

\item [1 pontos(s)] Como as metodologias ágeis, de modo geral,
encaram a fase de modelagem e especificação?

\item [1,5 ponto(s)] Como é realizada a implementação nas metodologias:
  Scrum, XP e Kanban? Explique como os requisitos são escolhidos para
  serem implementados e se a implementação é continua ou em intervalos
  de tempo.

\item [0,5 ponto(s)] Como é a relação com o cliente nas metodologias
ágeis quando comparadas com o Cascata? 

% 5. `[1,5 ponto(s)]` O desenvolvedor está sempre reavaliando suas
% metas nas metodologias Scrum, XP e Kanban. Em que momento ou qual
% a tarefa em que é feita esta reavaliação em cada uma das metodologias
% citadas?

\end{enumerate}

\question{2,0} Liste e descreva os ciclos de vida do software.

\question{3,0} Relacione as características mostradas na
Tabela~\ref{tab:met} de cada metodologia, descrevendo a principal
diferença quando comparada com as metodologias ágeis.

\begin{table}[ht]
  \centering
\begin{tabular}{l|l}\hline
  \bf Metodologia & \bf Característica \\\hline
  Cascata & Documentação \\
  Evolutivo & Protótipo \\
  Espiral & Iteração \\\hline
\end{tabular}
\caption{Relação das metodologias e algumas de suas características.}
\label{tab:met}
\end{table}

\question{3,0} Descreva a motivação para o surgimento das metodologias
ágeis o quais são seus valores?


%%%%%%%%%%%%%%%%%%%%%%%%%%%%%%%% BUFFER

% %%%%%%%%%%%%%%%%%%%%%%

% 1) Baseando-se no artigo https://www.ibm.com/developerworks/community/blogs/tlcbr/entry/mp208?lang=en[Gerência de Projetos Ágil ou PMBOK®?]
% e nas aulas, responda às questões a seguir:

% * Além do planejamento, quais os outros paradoxos ou diferenças (se houver) entre o PMBOK e a Gerência de Projetos Ágil, ligados a:
% ** `[1 ponto]` execução;
% ** `[2 pontos]` monitoramento/controle;
% ** `[0,5 ponto]` encerramento.
% * `[1,5 ponto]` Como os executores de um Projeto Ágil tentam minimizar os erros que possam ser introduzidos pela alteração frequente no escopo?
% * `[3 pontos]` Como é feita a estimativa de prazo e escopo (requisitos) em cada uma das metodologias ágeis a seguir?
% ** Scrum;
% ** _Extreme_ _Programming_;
% ** Kanban. 

% %%%%%%%%%%%%%%%%%%%%%%

% 2) `[2 pontos]` Baseando-se no artigo
% https://flavioaf.wordpress.com/2010/01/25/pmi-agile-desenvolvimento-agil-com-praticas-do-pmbok/[PMI
% Agile: Desenvolvimento Ágil com práticas do PMBoK?], explique o equívoco na comparação entre
% o PMBoK e os métodos de gerência ágil? Você concorda com o autor? Por quê?

% %%%%%%%%%%%%%%%%%%%%%%%%%%%%%%%%%%%%%%%%%%%%%%%%%%%%%%%%%%%%%%%%%%%%

