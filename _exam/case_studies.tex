\paragraph{1.} Com relação ao artigo \href{https://imasters.com.br/desenvolvimento/e-assim-que-o-facebook-desenvolve-software-e-o-coloca-em-producao-devo-me-preocupar/}{``É assim que o Facebook
desenvolve software e o coloca em produção. Devo me preocupar?''},
responda às questões:

\begin{enumerate}[a)]
\item Descreva a abordagem de desenvolvimento do Facebook. Está mais
  próxima da clássica ou ágil?
\item Kent Beck é citado no artigo, qual metologia ele foi criador?
\item O termo {\em Sprint} é citado no texto, em qual metodologia é utilizado?
  Descreva-o sucintamente.
\item Por quê o Facebook não realiza testes? Isto está de acordo com as
  metodologias ágeis? Explique sua resposta.
\item Ao invés da programação em pares, o Facebook adota a revisão do
  código. Quais as vantagens deste procedimento em termos de qualidade e
  custo do código?
\end{enumerate}

\paragraph{2.} Com relação ao artigo \href{http://computerworld.com.br/adote-metodologia-agil-para-viabilizar-transformacao-digital}{``Adote metodologia ágil para viabilizar a transformação digital''}, responda às questões:

\begin{enumerate}[a)]
\item No texto é citado a entrega frequente e adequação às mudanças de
  requisitos das metodologias ágeis. Isto ocorre com o uso das
  metodologias clássicas? Explique sua resposta.
\item Outro ponto citado é a utilização de testes, quais as diferenças
  entre os testes das metodologias ágeis e clássicas?
\item O autor avalia as práticas e processos das metodologias
  clássicas como burocráticas. Na sua visão, qual o motivo desta
  avaliação?
\end{enumerate}
 
\paragraph{3.} Com relação ao artigo \href{https://conteudo.startse.com.br/educacao/redacao/os-4-pilares-da-metodologia-de-trabalho-do-google/}{``Os 4 pilares da metodologia de
trabalho do Google''}, responda às questões:

\begin{enumerate}[a)]
\item A empresa Google baseia o desenvolvimento de sistemas em alguma
  metodologia ou possui sua própria? Descreva sucintamente como é a
  metodologia.
\item O artigo descreve que na empresa acontece reuniões semanais para
  acompanhamento do que acontece nos projetos. Faça uma analogia deste
  processo com algum processo de outra metodologia de sua escolha.
\item Qual é a métrica utiliza pela empresa?
\item Qual a importância do foco para a obtenção de resultados na
  empresa?  E qual o número máximo de projetos que um funcionário pode
  se envolver?
\item A avaliação de resultados do método OKR á parte integrante
  da metodologia? Explique sua resposta.
\end{enumerate}

\end{document}

\newpage

\begin{verbatim}

É assim que o Facebook desenvolve software e o coloca em produção. Devo me preocupar?

Por Jim Bird em	06/11/2013
  
 
Um recente artigo acadêmico publicado pelo prof. dr. Feitelson da
Hebrew University, por Eitan Frachtenberg, pesquisador no Facebook, e
por Kent Beck (que também está fazendo alguma coisa no Facebook),
descreve a abordagem do Facebook para desenvolver e colocar em
funcionamento o seu software frontend. Apesar de que seria muito mais
interessantes entender como o desenvolvimento backend é feito (é lá
que a maior parte do trabalho é realizado para manipular centenas de
milhões de usuários), há algumas coisas no artigo que valem a pena
saber.

Continuous Deployment no Facebook não é Continuous Deployment Em vez
de planejarem o trabalho em projetos ou dividir o trabalho em Sprints,
desenvolvedores do Facebook fazem a maior parte do trabalho em
pequenas mudanças independentes que são lançadas frequentemente. Isso
faz sentido no modelo de negócio online do Facebook, no qual todos
estão ajustando constantemente a plataforma e tentando novas opções e
aplicativos em diferentes comunidades de usuários e vendo o que
funciona. Isso se deve à arquitetura que permite que mudanças tão
pequenas e independentes possam na verdade ser feitas de maneira
independente e barata.

O Facebook diz que eles seguem o continuous deployment, mas não se
trata de continuous deployment da forma que IMVU tornou popular, na
qual todas as mudanças são enviadas aos clientes imediatamente, ou
mesmo como uma empresa como a Etsy faz continuous deployment.

No Facebook, código pode ser lançado duas vezes por dia, mas isso é
normalmente feito principalmente para correções de bugs e código
interno. Código de produção novo é lançado uma vez por semana:
milhares de mudanças de centenas de desenvolvedores são empacotadas no
domingo pela pequena equipe de “release” que eles possuem, passadas em
testes automatizados de regressão e colocados em funcionamento na
terça-feira, caso os desenvolvedores que contribuíram com as mudanças
estiverem presentes. Engenheiros de “release” avaliam o risco das
mudanças baseados no tamanho delas, na quantidade de discussão feita
na revisão do código (que é gravada através de um sistema interno de
revisão de código) e no “push karma” de cada desenvolvedor: em quantos
problemas eles viram no código desse desenvolvedor anteriormente.

Uma ferramenta chamada “Gatekeeper” (porteiro) controla que recursos
estarão disponíveis para quais clientes de forma a suportar dark
launching, e todo código é disponibilizado incrementalmente – colocado
em “staging”, depois para um grupo de usuários, e assim por
diante. Mudanças podem ser desfeitas caso seja necessário –
individualmente, ou como último recurso, todo um release de
código. Entretanto, como em muitas empresas do Vale do Silício que
adotam devops, eles seguem o lema “homens de verdade só vão para
frente”.

Propriedade de código Uma chave para a cultura do Facebook é que os
desenvolvedores são individualmente responsáveis pelo código que eles
escrevem, para testá-lo e suportá-lo em produção. Isso está refletido
no modelo de propriedade de código que eles adotam:

Desenvolvedores precisam dar suporte para o uso operacional de seu
código – uma combinação que vem sendo chamada de “devops”. Isso
incentiva a escrita de código de qualidade e testes extensivos. O
risco pessoal dos desenvolvedores em manter o sistema funcionando bem
complementa os procedimentos de engenharia e permite que o sistema
mantenha qualidade em escala. Metodologias e ferramentas não são
suficientes porque elas sempre podem ser mal utilizadas. Por isso, uma
cultura de responsabilidade pessoal é algo crítico.

Consequentemente, a maior parte dos arquivos de código fonte é
modificada por apenas alguns engenheiros. Apesar de pelo menos mais um
engenheiro rever todas as mudanças antes que sejam feitos os commits,
um terço dos arquivos de código fonte foram editados por apenas um
engenheiro e outro quarto por dois. Apenas 10\% dos arquivos foram
manipulados por mais de 7 engenheiros. Por outro lado, a distribuição
dos engenheiros pelos arquivos possui uma longa cadeia, com os
arquivos mais compartilhados sendo utilizados por não menos de 870
engenheiros distintos. Esses arquivos amplamente compartilhados são
predominantemente arquivos de bibliotecas e também incluem principais
configurações e arquivos de alto nível do PHP.

Testes? Não precisamos de testes… O Facebook não possui um time de
testes independente, porque eles dizem que não precisam de um.

Primeiro, eles dependem muito de revisão de código para encontrar os bugs:

No Facebook, revisão de código ocupa uma posição central. Cada linha
de código é escrita e revisada por um engenheiro diferente do autor
original. Isso serve a múltiplos objetivos: o engenheiro original é
motivado a assegurar que seu código é de alta qualidade, o revisor
chega com uma “cuca fresca” e talvez encontre defeitos ou sugira
alternativas e, em geral, o conhecimento sobre as práticas de
programação e do próprio código se espalham pela empresa.

Desenvolvedores também são responsáveis por escrever testes unitários
e seus próprios testes de regressão – eles possuem “dezenas de
milhares de testes de regressão” (o que não parece ser suficiente para
as mais de 10 milhões de linhas de código em sua maioria em PHP
compilado para C++, ambas linguagens nas quais é comum comentar erros)
e testes automatizados de desempenho.

E os desenvolvedores também testam o software ao usar a versão de
desenvolvimento do Facebook em suas próprias contas pessoais da rede
social. De acordo com os autores “este é apenas um aspecto do abandono
das técnicas tradicionais de desenvolvimento de software”. Mas
desenvolvedores do Facebook usando seus próprios softwares
internamente (e considerando isso como “teste”) não é diferente do
passado da Microsoft, em que os empregados supostamente deveriam
“provar do próprio veneno”, uma prática que surtiu pouco, ou nenhum
efeito, na qualidade dos produtos da empresa.

O Facebook também depende dos clientes para testar seu software para
eles. O software é lançado em etapas para testes A/B e “experiência em
tempo real”, em subgrupos da base de usuários, caso os clientes
queiram participar desses testes ou não. Pelo fato de a base de
usuários deles ser tão grande, eles podem ter um feedback
significativo a partir dos testes, mesmo com uma pequena porcentagem
dos seus usuários, o que pelo menos minimiza o risco de inconvenientes
para os clientes.

Segurança???

Apesar de o desempenho ser considerado importante para os
desenvolvedores do Facebook, não há menção a verificações ou testes de
segurança em lugar algum da descrição de como os devops da empresa
colocam software para rodar. Sem análise/scanner estático ou dinâmico,
teste de invasão ou explicação de como o time de segurança e
desenvolvedores trabalham juntos, nem mesmo para “código relacionado à
privacidade” – apesar de esse código ser “mantido em padrões mais
elevados”, eles não explicam o que são “padrões mais
elevados”. Presumivelmente, eles contam com o uso de bibliotecas e
frameworks para lidar pelo menos com alguns aspectos de segurança no
aplicativo e possivelmente procurar por bugs de segurança em suas
revisões de código, mas eles não dizem nada.

Não há muita informação disponível sobre o programa de segurança do
Facebook. O time de segurança da empresa parece gastar bastante tempo
educando as pessoas sobre como usar o Facebook de forma segura, como
desenvolver aplicativos seguros e executando seu programa de caça aos
bugs, que paga uma recompensa em dinheiro a pessoas de fora da
companhia que foram capazes de encontrar um bug para eles.

Uma busca sobre segurança no Facebook em sua maior parte retorna uma
longa lista de falhas públicas de segurança, violações de privacidade
e vulnerabilidades de segurança encontradas em todos esses anos e que
continuam até hoje. Talvez a falta de um programa efetivo de segurança
(appsec) seja a causa para isso.

Essa é a forma como o Facebook desenvolve. Devo me importar?  Apesar
de ser interessante ver como funciona por dentro uma empresa tão
conhecida como o Facebook e como eles desenvolvem software em grande
escala, não fica claro por que esse artigo foi escrito. Há pouco sobre
o que o Facebook está fazendo (em seu desenvolvimento frontend, pelo
menos), algo que seja exclusivo ou inovador, exceto talvez pela forma
como eles usam BitTorrent para enviar código para os milhares de
servidores, como faz o Twitter, algo que eu já tinha ouvido há alguns
anos na Velocity e que já havia sido escrito antes.

Eu gosto da ideia de desenvolvedores serem responsáveis por seu
trabalho, desde o início até a produção, que é um princípio que também
seguimos. Revisões de código são boas. Recursos de dark launching são
uma boa prática e têm sido comum há um bom tempo (mesmo antes de serem
chamadas de “dark lauching”). Não ter testes para appsec não é bom. De
qualquer forma, não sei o que podemos aprender ou gostaríamos de usar
a partir da experiência deles.

***

Artigo traduzido pela Redação iMasters, com autorização do
autor. Publicado originalmente em
http://swreflections.blogspot.com.br/2013/09/this-is-how-facebook-develops-and.html
\end{verbatim}


\newpage

\begin{verbatim}
Adote metodologia ágil para viabilizar a transformação digital

Entregar frequentemente algo na menor escala de tempo faz com que o
cliente consiga ver o valor agregado e o progresso do produto

João Gatto*
23 de Junho de 2015 - 08h30

Ultimamente muito se tem falado de transformação digital, conceito que
vai além de automatizar processos, e mostra que os modelos de negócio
deixaram de ser algo imobilizado, ou seja, modelos de negócio devem
ser criados e alterados de maneira contínua. E para que isso fosse
possível, muitas frentes como Cloud Computing, Big Data, Mobile
Payment, IoT (Internet of Things), entre outras, foram criadas para
inovar e melhorar os serviços e produtos fornecidos pelas
empresas. Mas será que as empresas estão adotando maneiras ágeis e
flexíveis para fornecer tais serviços?

Devido à rápida evolução tecnológica, as empresas estão se
transformando e criando concorrência inesperada para os gigantes do
mercado. Por exemplo, recentemente tivemos a criação do Uber na
indústria de táxis, Airbnb na indústria hoteleira, Netflix na
indústria de filmes e entretenimento, Whatsapp na indústria da
comunicação, entre outros que influenciam diretamente os modelos de
negócios que antes estavam estagnados e que foram alterados usando o
que a tecnologia tem a oferecer atualmente como forma de se posicionar
novamente de maneira competitiva no mercado.

 Por isso muitas empresas estão adotando a metodologia ágil na sua
maneira de trabalhar e estão vendo que ela não é apenas uma melhoria
de processo, mas sim uma nova necessidade estratégica. Quando se trata
de metodologia ágil, o grande desafio é criar o mindset ágil. Não
significa apenas ter as skills e know-how, mas também cultivar um time
ágil, sendo suportado pela alta gerência da organização.

Diante desse cenário de mudança constante, como a metodologia ágil
pode ajudar na entrega da transformação digital? Simples: ao contrário
das metodologias tradicionais, que demandam muito tempo para entregar
algo funcionando ao cliente, um dos princípios da metodologia ágil é
satisfazer o cliente através de entregas em curtos espaços de tempo de
maneira contínua e adiantada, sempre tendo em mente o mais importante:
o valor agregado.

Entregar frequentemente algo na menor escala de tempo, entre duas a
três semanas, faz com que o cliente consiga ver o valor agregado e o
progresso do produto. Quando entregamos valor agregado precisamos ter
em mente três pontos: custo (quanto custa fazer esta parte do
produto?), negócio (qual o retorno esta parte do produto trará para o
negócio como um todo?) e o ROI (quanto tempo vai levar até obter o
retorno sobre o investimento?). Agora imagine um projeto para mobile
payment, seguindo a metodologia tradicional, na qual o cliente só
receberia o produto pronto, de acordo com o es

copo acordado depois de alguns meses. Será que os requisitos
levantados pelo cliente são os mesmos que foram acordados no início do
projeto? Será que as tecnologias para esse tipo de projeto não
sofreram alguma alteração durante esse período?

Usando a metodologia ágil há um melhor gerenciamento da mudança de
requisitos, pois ela assume que os requisitos iniciais vão ser
alterados, mesmo tardiamente, podendo criar uma vantagem competitiva
para o cliente e ainda ser um diferencial para a empresa prestadora do
serviço.

Outro ponto importante quando falamos de metodologia ágil é a questão
da qualidade, que é garantida através da correta aplicação das
melhores práticas, nas quais são realizados frequentes testes em cada
uma das funcionalidades, possibilitando a identificação de qualquer
problema com antecedência necessária para que a entrega final do
produto seja feita dentro do prazo e das especificações acordadas com
o cliente.

A transformação digital hoje é uma realidade que vem crescendo e
revolucionando o mercado de TI. As empresas que ainda usarem
metodologias de desenvolvimento e entrega com práticas e processos
burocráticos estão fadadas a perder mercado para os concorrentes. Para
evitar que isso aconteça, novos processos funcionam como aliados e
devem ser adotados por todos da empresa.

*João Gatto é Technical Analyst do Agile Center da GFT Brasil,
companhia de Tecnologia da Informação especializada no setor
financeiro.

\end{verbatim}

\newpage

\begin{verbatim}
Os 4 pilares da metodologia de trabalho do Google

Com milhares de funcionários, gigante da internet faz questão de
manter a cultura e a organização de uma startup


Perto de completar vinte anos, o Google já deixou de ser startup há
mais de dez – mais precisamente em 19 de agosto de 2004, quando
ocorreu sua abertura de capital – mas nem por isso deixou de atuar
como tal.

Tanto a cultura organizacional como a gestão da empresa ainda seguem
os mesmos princípios das melhores e mais enxutas empresas de
tecnologia do Vale do Silício, nos EUA, algo definido internamente
como “Google Style”.

Quem diz isso é José Papo, gerente de relações com startups e
developers do Google na América Latina.

Segundo Papo, manter a atuação como startup em meio ao grande
crescimento é algo muito difícil. “Chega um momento em que não se
mantém mais esse ritmo, mas o Google e outras empresas do Vale
conseguem”, afirma. “E o que todas elas têm em comum é o método de
organização”, complementa.

O método é levado tão a sério que o Google faz questão de replicá-lo
em todas as startups em que investe (só em 2014 a empresa investiu
mais de R$ 1 bilhão, leia aqui). “Nosso investimento não é apenas
financeiro, mas também cultural”, avalia Papo.

Conheça a seguir os quatro pilares dessa metodologia e comece a
aplicá-los hoje mesmo na sua startup. Eles podem ajudá-lo no que há de
mais importante em uma empresa, a execução, como conclui o gerente do
Google: “Só a ideia não vale muito, o que vale é a execução dessa
ideia. E a execução é feita com método.”

1. Liberdade com responsabilidade

Uma das coisas que o Google mais valoriza são as pessoas. É por isso
que muitos dos benefícios de trabalhar na empresa são conhecidos e
amplamente difundidos, como a liberdade de horários, o ambiente
descontraído de trabalho e outros.

Segundo José Papo, a multinacional considera que o maior ativo das
empresas de tecnologia são os chamados “criativos inteligentes” (smart
creatives). “Eles não são só os desenvolvedores de software, são todos
que trabalham criando coisas, de qualquer área. São engenheiros,
publicitários, marqueteiros…”, explica.

Dessa consciência nasce a primeira premissa do Google: para trabalhar
com pessoas extremamente criativas e inteligentes, você precisa ter
liberdade de atuação. “Toda a nossa organização se baseia nesse
primeiro princípio, que é liberdade com responsabilidade”, define o
gerente.

2. Objetivos e resultados-chave

Outro pilar da gigante da internet é método de gerenciamento das
atividades dentro da empresa, chamado OKRs, Objectives and Key Results
(objetivos e resultados-chave). Também adotado por diversas startups
do Vale do Silício, é baseado na simplicidade e na disciplina.

“O método OKR define a forma como as áreas e cada indivíduo
trabalham”, explica Papo. Mas isso não quer dizer que é um plano
rígido e complexo. “O Google não gosta disso. É um método simples,
usamos o modelo ‘lean startup’ (startup enxuta). Não ficamos
escrevendo páginas e páginas de coisas para desenvolver nossas
ideias. Somos enxutos e focados”, completa.

A disciplina é outro ponto importante do OKR, ajudada pela comunicação
aberta de todos os processos. “Não adianta falar ou escrever o que
você quer atingir se não há acompanhamento. Existe uma intranet no
Google onde escrevemos nossos OKRs e todo mundo sabe o que todo mundo
está fazendo. Os objetivos e resultados são trimestrais e são públicos
para todos os funcionários. Eu sei quais são os OKRs do Larry Page (um
dos fundadores do Google)”, exemplifica Papo, para quem o princípio de
“quanto mais informação aberta, melhor” também ajuda também na
cooperação entre tarefas.

O acompanhamento é feito por meio de reuniões semanais de
resultados. Até mesmo os encontros dos principais nomes da empresa são
abertos, transmitidos ao vivo via streaming e gravadas, para que
qualquer funcionário assista quando quiser. Essa cultura, uma surpresa
no início, é hoje elogiada por Papo: “Toda semana sabemos o que
acontece na empresa, quais são as prioridades. A grande maioria das
grandes empresas não faz isso. Elas tendem a esconder informação, não
abrir”.

3. Métricas

 O terceiro item do método de trabalho Google são os números. Nas
 palavras de José Papo, a cultura orientada a métricas faz com que o
 Google não seja regido pelos “hipopótamos”, que no chavão empresarial
 são as pessoas mais bem pagas da organização.

“Normalmente, nas grandes empresas é assim: quando o vice-presidente
fala, todo mundo fica quieto. No Google não é assim. Até um
estagiário, se ele tiver os números para embasar sua opinião, mesmo
que seja contra a opinião de um vice-presidente, ele pode questionar”,
exemplifica o gerente. Ele diz ainda que valorizar as pessoas que
pensam diferente garante a inovação nas empresas: “Para ser uma
empresa inovadora você tem que ter pessoas inteligentes com poder de
questionar embasados por números.”

4. Foco

O último pilar do Google é o que José Papo diz ser o principal ponto
de aprendizado de uma startup desde o início: foco. “Se você não tiver
foco, vai morrer. Não tente fazer vinte, trinta coisas ao mesmo
tempo. Foque no que que realmente é importante ou crítico para aquele
momento que você vive”, defende.

E é justamente para manter o foco e por ser de simples e fácil uso que
o Google adota o método OKR. Definidos os períodos de tempo, são
traçados os objetivos, que podem ser qualitativos ou quantitativos, e
os itens necessários para alcançá-los (resultados-chave). Estes
últimos precisam ser sempre quantitativos, porque precisam ser
mensurados.

Pelas regras do Google, cada funcionário pode ter no máximo cinco
objetivos por trimestre, com quatro resultados-chave dentro de cada
objetivo. “Mais do que cinco significa para nós que você está fazendo
coisas demais, não está focado”, justifica. Mas isso depende de cada
empresa, há quem trabalhe com períodos menores ou maiores.

No final do período, os resultados são avaliados pelo próprio
funcionário que executa o OKR e, num segundo momento, em conjunto com
o líder, para que ajustes sejam feitos. E isso acontece em todos os
níveis da empresa, cada um com seus OKRs, departamentos, times e
indivíduos. “Não é um método usado para avaliação de performance. Essa
interação ocorre para aumentar a inovação e a produtividade”, finaliza
Papo.

Livros para saber mais:

“Como o Google Funciona” (320 páginas), de Eric Schmidt (CEO do
Google) e Jonathan Rosenberg. Reúne as lições que fizeram da empresa
uma gigante global conhecida pela missão de inovar
incessantemente. Eles descrevem como o avanço da tecnologia transferiu
o poder das corporações para os consumidores e reforçam que, para
sobreviver, é essencial concentrar esforços na qualidade dos produtos
e investir em uma nova categoria de profissionais – os criativos
inteligentes, que unem conhecimento técnico, tino comercial e uma
criatividade sem limites.

“Google, a biografia – Como o Google pensa, trabalha e molda nossas
vidas” (464 páginas), do jornalista Steven Levy. Analisa como o Google
conseguiu ser uma das poucas corporações na história a ser tão
bem-sucedidas e admiradas quanto, transformando a internet e se
tornando parte indispensável de nossas vidas. Este livro revelador
leva os leitores para dentro dos escritórios do Google e mostra como a
organização funciona.


\end{verbatim}