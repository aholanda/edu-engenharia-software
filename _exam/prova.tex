
%%% Local Variables: 
%%% mode: latex
%%% TeX-master: "prova"
%%% End: 

\documentclass[10pt]{../prova}

%\usepackage{pgf-umlcd}
\usepackage{listings}
\lstset{language=java,basicstyle=\footnotesize}

\begin{document}

%\disciplina{``Engenharia de Software II''}
\disciplina{``Fundamentos de Sistemas de Informação II''}
\curso{Sistemas de Informação}
\data{27/9/2016}
\makehead




\end{document}



\end{document}



%\section*{Projeto Orientado a Objetos}


\input{../oop/exercises/classes-t}

\input{../oop/exercises/polymorphism-t}


\end{document}



\end{document}



\question{6} Após levantamento de requisitos para o Hospital Dante Divino,
verificou-se a necessidade da modelagem/especificação das seguintes
funcionalidades:

\begin{enumerate}[a)]
\item Cadastro e gerenciamento dos dados da paciente;
\item Agendamento de exames;
\item Consulta de exames realizados;
\item Agendamento de consultas;
\item Consulta das informações das consultas.
\end{enumerate}

\medskip Tomando como base os requisitos levantados, desenhe:

\begin{enumerate}[a)]
\item O(s) diagrama(s) de fluxo de dados (DFD).

\item O diagrama entidade-relacionamento (DER) para o modelo de dados do
sistema a ser implementado.

\item O diagrama de classes para as entidades identificadas no modelo de dados, com
os métodos ou operações essenciais que cada classe pode executar.

\end{enumerate}

