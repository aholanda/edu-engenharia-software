3) `[1,5 ponto(s)]` Vamos supor que queiramos adicionar os arquivos
`RSA.java` e `TestRSA.java` ao repositório do controle de versão
`git`. Liste os comandos necessários para:

1. Adicionar os arquivos para o `git` controlar;
2. Consolidar o código ou alterações efetuadas;
3. Enviar o código ao repositório;
4. Baixar as alterações do repositório;
5. Verificar o estado (_status_) da cópia local do repositório;
6. Marcar um estado do projeto de software, que pode ser por exemplo,
   um versão lançada.


`[1 ponto(s)]` Em um sistema de controle de versão distribuído, dois
   desenvolvedores podem ter suas cópias locais do projeto e enviar
   as modificações quando for conveniente para o repositório central.
   O que acontece quando os dois desenvolvedores modificam o mesmo
   trecho do código e o desenvolvedor A envia primeiro que o B?
   O que deve ser feito para o desenvolvedor B adicionar suas
   alterações sem risco de se perder?

4) `[1 ponto(s)]` Quais as diferenças entre o GNU `make` e o Apache `Ant`?

5) `[1 ponto(s)]` Para os fragmentos de código a seguir, complete os campos
   `__` com os valores corretos.

[source,java]
----
public class RSA {
       public int totem(int p, int q) {
       	      return (p-1)*(q-1);
	}
}

public class TestRSA {
       public int testTotem() {
       	      RSA rsa = new RSA();
	      int esperado = 12;
	      int resultado = rsa.totem(3, __);

	      assertEquals("o totem de 3 e __", esperado, resultado);
       }
}
----

6) `[1,5 ponto(s)]` Como o Sistema de Gerenciamento de _Bugs_ pode ser usado
para gerenciar as tarefas de Gerenciamento de Projetos:

* Planejamento;
* Execução;
* Monitoramento/Controle;
* Encerramento.

Responda tendo em mente um projeto de desenvolvimento de software.

7) `[1 ponto(s)]` O que pode ser feito para que toda a equipe não pare quando a
montagem ou os testes durante a integração contínua falharem?

////

\question{0,5}~Assinale a alternativa que possui a tarefa que {\bf não}
faz parte do Gerenciamento de Configuração:

\begin{enumerate}[a)]
\item Controle de versão.
\item Automação da montagem do sistema.
\item Levantamento de requisitos.
\item Automação dos testes e rastreio de bugs (bug-tracking).
\item Controle do lançamento de versões.
\end{enumerate}

