\question{3} Para o desenvolvimento de um ``Módulo de Controle de
Notas de Alunos e Gerenciamento de Conteúdo'' foram levantados os
seguintes requisitos: armazenamento de conteúdo de aulas, acesso à
internet, acesso ao material didático, registro de dados da
disciplina, portabilidade (multiplataforma), controle de acesso do
usuário, acessibilidade, cadastro de pessoas (alunos, docentes e
funcionários), responsividade (mesmo layout em diferentes
dispositivos) e desempenho.

\noindent Com base nos requisitos apresentados classifique-os em
funcionais e não-funcionais.

\question{}~Quais as diferenças entre as metodologias/processos
Cascata, Espiral e RUP durante a fase de implementação?

\question{}~Descreva como o levantamento de requisitos é influenciado
pelo metodologia/processo escolhido. Como os requisitos são representados
nas seguintes metodologias?
\begin{itemize}
\item RUP;
\item XP;
\item Scrum;
\item Kanban.
\end{itemize}


\paragraph{Descrição geral.}~Quais funções o software irá desempenhar?
Descrever em alto nível, sem detalhes técnicos. O publico-alvo é o
cliente.

\paragraph{Descrição técnica.}~Quais os componentes necessários para
desempenhar a função desejada? Descrever em alto nível, documento para
o desenvolvedor.

\begin{itemize}
\item Interface com o usuário: gráfica, texto, sem interface. Listagem
  dos componentes básicos.
\item Requisitos de hardware;
\item Interface de comunicação: camadas da rede de computadores;
\item Requisitos de memória: memória primária, memória secundária,
  armazenamento externo;
\item Tipo de sistema: web, aplicativo desktop ou móvel, servidor;
\item Necessidades de segurança;
\item Portabilidade: em quais plataforma irá rodar;
\item Confiabilidade: manutenção da integridade dos dados;
\item Disponibilidade: uso intensivo, ocasional, horário de trabalho;
\item Restrições: tipos de dados devem ser controlados, tais como datas e finanças;
\item Performance;
\item Meio de armazenamento: banco de dados, arquivos.
\end{itemize}

