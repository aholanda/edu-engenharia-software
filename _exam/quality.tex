\paragraph{Exercício \exno.} Deming ensinou que a qualidade deve estar é
conformidade ao processo, enquanto Crosby insistia que é em conformidade à
especificação. Como estas duas formas de abordagens afetam a produção de
sistemas de informação?

\paragraph{Exercício \exno.} Por quê a norma ABNT NBR ISO 9001:2015 é
difícil de ser implementada na prática? Quais as consequências deste
fato com relação a sua utilização e certificação das empresas?

\paragraph{Exercício \exno.} Descreva as principais diferenças entre o
CMMI e o SPICE quanto:

\begin{enumerate}[a)]
\item Modelo de referência;
\item Às dimensões e níveis (capacidade e maturidade);
\item Mecanismo de pontuação;
\item Locais de maior implantação (países).
\end{enumerate}

\question{1,5} Quais as principais diferenças entre ``conformidade à
especificação'' e ``conformidade ao processo'' no contexto de
desenvolvimento de software? E entre metodologia de desenvolvimento de
software e modelo de processo de produção de software?

\question{1,5} Um dos pontos-chave da teoria de William E.\ Deming era
``o controle estatístico da qualidade''. O conceito de Feigenbaum
tinha como diretriz ``um contínuo processo de melhoria''. A melhora de
qualidade no sistema de Juram tinha como essência ``a identificação
dos pontos de melhoria e posterior aperfeiçoamento dos processos de
produção''. Como estes princípios se relacionam com os níveis de
maturidade do CMMI?

\question{1,0} O que são dimensão do processo e dimensão de capacidade
do processo no modelo SPICE? 


\question{2} O trecho a seguir foi extraído do artigo 
``Expandindo o Agile para se adequar ao CMMI Nível 3'' de 
David J. Anderson -- Microsoft Corporation:

\begin{quote}
  Em agosto de 2004, nós (na Microsoft) começamos a desenvolver uma
  metodologia para o novo produto, Team System, que seria compatível
  com o CMMI (Capability Maturity Model Integration) do SEI (Software
  Engineering Institute) [SEI 2002]. Nesse momento, não imaginamos que
  a solução final seria vista como um método ágil. Assumimos que o
  mundo do CMMI e o mundo ágil eram como óleo e água. O CMMI e seu
  antecessor, o Software CMM [SEI 1993], foram amplamente associados a
  negócios de contratação, aeroespaciais e de defesa do governo--um
  mundo de programas grandes e de longa duração, frequentemente em
  sistemas importantes para a vida, com exigências governamentais de
  auditabilidade e rastreabilidade.
\end{quote}

Quais as diferenças entre o CMMI e as metodologias ágeis que levaram o
autor do texto afimar que os dois são como ``água e óleo''?

% https://msdn.microsoft.com/pt-br/library/cc517970.aspx
\question{2,5} O trecho a seguir foi extraído do artigo 
``Expandindo o Agile para se adequar ao CMMI Nível 3'' de 
David J. Anderson -- Microsoft Corporation:

\begin{quote}
  O CMMI é um modelo para melhoria do processo que tem por objetivo
  levar uma organização a um nível em que o aprimoramento contínuo em
  produtividade e qualidade seja possível [Chrissis 2003]. Essa
  filosofia básica é baseada no trabalho de W.\ Edwards Deming [Deming
  1982]. Entretanto, o modelo de 5 níveis que foi apresentado pelo CMM
  de Software original para permitir que a avaliação da maturidade em
  empreiteiras governamentais para os contratos da Força Aérea dos
  Estados Unidos fosse baseada n modelo de fabricação de Philip Crosby
  [1979] O nome de Crosby é mais associado à definição de qualidade
  como conformidade é especificação [Crosby 1979]. Portanto, para
  mensurar a qualidade, alguem deve ter uma especificação completa no
  início e no final a variação derivada da especificação é mensurada
  e a qualidade é avaliada. Para projetos de software, o sucesso é
  mensurado com a utilização da métrica de qualidade, interpretada
  como entrega de um projeto no momento certo, no devido orçamento,
  com a funcionalidade acordada. A influência de Crosby no CMM de
  Software é tão proeminente que não é amplamente apreciado que a
  filosofia e meta subjacentes do CMMI é alcançar aprimoramento
  contínuo como descrito e ensinado por Deming.  O modelo de
  maturidade do CMMI é mapeado ao pensamento de Deming. Os níveis 2 a
  4 se referem À criação da capacidade organizacional para eliminar a
  variação de causa especial e, portanto, evitar erros de
  gerenciamento n.\ 1 e n.\ 2, enquanto o nível 5 fornece
  aprimoramento contínuo por meio da redução gradual da variação de
  causa comum. Se o CMMI estivesse realmente enraizado na filosofia de
  Deming, nos pareceria que deve ser possível criar um método CMMI
  realmente ágil.
\end{quote}

Quais foram os princípios propagados por Deming e Crosby que o autor
afirma que influenciaram o CMMI? Descreva a relação destes princípios
com a classificação dos critérios de capacidade e maturidade do CMMI.

\question{2,5} A Tabela~\ref{tab:cmmi-spice} apresenta trecho de uma
tabela existente no livro ``Automotive SPICE in Practice: Surviving
Implementation and Assessment'' escrito por Markus Mueller, Klaus
Hoermann, Lars Dittmann e Joerg Zimmer. Os autores relacionam empresas
e o processo de produção adotato.

\begin{table}[ht]
  \centering
  \begin{tabular}[ht]{l|l|l}\hline
    \bf\hfil Empresa &\bf\hfil Divisão &\bf\hfil Processo \\\hline
    BMW & Cadeia de processos E/E & CMMI + processo próprios\\\hline
    Chrysler & Núcleo de Engenharia E/E & CMMI/SPICE\\\hline
    Mercedes & EUA & CMMI \\
    Grupo Automotivo & Alemanha & SPICE\\\hline
    Toyota & & Proprietário baseado no CMMI\\
    Bosch Automotive & Unidades de produção de software & CMMI/SPICE\\\hline
    Continental Sistemas & Todas unidades & SPICE\\
    Automotivos & que produzem software & \\\hline
    Delphi & Eletrônica e Segurança & CMMI \\\hline
    Siemens VDO & Todas as 13 divisões & CMMI \\\hline
    Valeo & Todas divisões & CMMI/SPICE \\
    & que produzem software & \\\hline
  \end{tabular}
  \caption{Tabela de utilização dos modelos CMMI e SPICE em algumas empresas.}
  \label{tab:cmmi-spice}
\end{table}
{\small\noindent E/E: Elétrica/Eletrônica}

Baseando-se na Tabela~\ref{tab:cmmi-spice}, responda às seguintes questões:
\begin{enumerate}[a)]
\item {\tt [0,5]} Como está dividida a utilização dos processos CMMI e
  SPICE de acordo com a localização geográfica das empresas?
\item {\tt [2]} Algumas empresas utilizam o CMMI e SPICE
  conjuntamente. Qual são as características em comum entre os dois
  processos que permitem este procedimento?
\end{enumerate}

\question{0,5} O ciclo PDCA ({\em Plan-Do-Check-Act}) proposto por Deming e Shewhart é composto pelos elementos:

\begin{enumerate}[a)]
\item Planejamento, Execução, Verificação, Ação.
\item Planejamento, Ação, Execução, Verificação.
\item Planejamento, Execução, Ação, Verificação.
\item Planejamento, Execução, Manutenção, Ação.
\item Planejamento, Ação, Verificação, Execução.
\end{enumerate}
